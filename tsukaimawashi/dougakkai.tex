
\subsecnomaru
%\subsectionを◯なしに

\section{全学学生自治会同学会のご紹介}
\label{sec:dogakkai}

\bunsekisha{文責}{京都大学全学学生自治会同学会再建準備会}

\subsection{0.はじめに}
受験生の皆さん、お疲れ様です。この文章を読んでくださっている全ての方、ありがとうございます。この文章は熊野寮に入る人にも入らない人も読んでください!\index{どうがくかい@同学会|textbf}

\emphbf{京大に入ると、皆さんは自動的に同学会員になります!!}「同学会」は、京都大学の全学部生が加盟する自治会です。高校までなら「生徒会」と呼ばれていた組織に似ていますね。以下で詳しく解説するので、入学したら積極的に同学会の活動に参加して学生自治を盛り上げていきましょう!

\subsection{1.「同学会」ってなに?}
長く言えば、「京都大学全学学生自治会同学会」。1946年に設立された京都大学の\emphbf{全学部生が所属する自治会}です。\index{がくせいじちかい@学生自治会}

「大学」と言った時に、そこには学部生の他にも、大学院生\index{だいがくいんせい@大学院生}、聴講生、教員、職員、清掃員、理事など、沢山の立場の人が含まれます。そして、それぞれの立場にはそれぞれの立場に応じた、要請や利害があります。その中で、\emphbf{学生の立場}をとことん追求するのが学生自治会です。京都大学には学生自治会がいくつかあります。各学部には学部自治会があったり、各寮には寮自治会があったり・・・。それの全学バージョンが同学会です。(大学院生は同学会員ではありませんが、オブザーバーとして同学会の運動に協力してくれている仲間が沢山います。「院生協議会」を復活させるという話もあります。)

\subsection{2.大学当局と仲が悪いの?}
\emphbf{仲悪いです。}でも以下を読めばその理由に納得してくれると思います。

現在の京都大学は理事会の独裁下にあります。8人の理事たち、しかもそのうち半数は現場もろくに知らない産業界の資本家たち、そんな人たちが京都大学の意思決定権の全てを握っています。敗戦直後の日本では、大学の戦争協力への痛烈な反省から\index{せんそうとがくもん@戦争と学問}、大学の自治ということが言われて、教授会や学生自治会などの大学を構成する当事者による運営が多少なりとも成立していました。しかし、今日ではそれがどの大学でも全部破壊されて、独裁が敷かれています。さらに悪いことに、理事会などの支配者層が目指している大学像と学生が目指している大学像は正反対といっても良いくらいに対立しているのです。すなわち、理事会は経営界にひたすら奉仕する大学を目指し、大学の就職予備校化を急激に進めてきたのです。その中で学生はGDP創出の道具にされ、主体的な活躍の場は無くなっていくでしょう。

このような大学の運営形態は、政府が主導する\emphbf{新自由主義的大学改革}のなかで加速してきました。大学改革によって、国から国立大学に降りる運営交付金は年々減少し、大学自身が経営体として稼ぐ体制が追求されています。そんな中で、\emphbf{大学改革の矛盾の全てを押し付けられているのは学生です}。学費は年々上がり、安い宿舎はアパートに代わり、学内の診療所は廃止され、言論の手段たる立て看板\index{たてかん@タテカン}は撤去され、カリキュラムが学生の声を無視して改悪され、学内で以前はできていたことが一つずつできなくなっていく・・・。\index{だいがくとうきょく@大学当局!によるだんあつ@---による弾圧}

\emphbf{そんな大学当局に対して、学生の立場を代表して闘うことこそが同学会の使命です!}戦前も戦後も学生自治会は何度も潰されかけました。しかし、その度に学生が建て直してきました。大学当局や国家権力と決定的に対立していても、何十年も闘い続けることのできる団結を生んできたのが同学会です。

\subsection{3.どうやって運営してんの?}

\emphbf{同学会規約に則って運営しています}。同学会には「代議員会」という最高議決機関が設けられています。全学から代議員が基本的には立候補制で選ばれ、運営しています。この文章を書いている時点で直近の代議員会は7月に開かれました。残念ながら、規約にある議決の定足数に満たなかったため、正式な代議員会として成立させることは叶いませんでしたが、結成された「同学会再建準備会」には新たな参加が続々と増えており、今後の活動への期待が高まりました。そこに集まった学生が有志として日常的な同学会の運営を担っています。このパンフレットが配られている頃には、2月の代議員会がすでに終了しているはずです。

\subsection{4.「全処対」ってなに?}\index{ぜんがくしょぶんたいさくいいんかい@全学処分対策委員会|textbf}
長く言えば、「京都大学全学学生自治会同学会全学処分対策委員会」。\emphbf{処分問題に取り組む同学会の組織}です。同学会に所属する有志によって2021年に設立されました。全処対は熊野寮自治会の対処分戦略推進局(処分局)など他の学生自治会とも連携して処分問題に取り組んでいます。処分問題については、このパンフレットの他の記事にも紹介されていますね。端的に言えば、大学に歯向かう学生を処分という卑怯な手で黙らせるという大学当局の攻撃の問題です。京大には様々な学内問題が蓄積しており、学生がそれを批判し、抗議することは大学の運営を担う一主体として当然です。しかし、当局によって処分されてしまう現状ではどんなにおかしくても声をあげられなくなる一方です。\emphbf{処分問題が解決しなければ、他の一切の学内問題に声を上げられなくなる}という意味で他の問題とは一つ次元が違う問題であると捉えて、処分問題に専門的に取り組む組織ができました。\index{がくせいしょぶん@学生処分}

デモや学内での集会、カンパ集めなどを運動の中心に据えて、戦闘的に活動してきました。数年の運動は、学内の集会において、処分を受けて放学にされた挙句、京都大学内立ち入り禁止にされた仲間を実力で構内に入構させ、一切の新たな処分をさせないくらいの団結を生み出しています。そのくらい強くて、明るい展望をもった組織だということです。

\subsection{5.「二つの執行部問題」ってなに?}
同学会には、執行委員会という執行部があります。その執行委員会が現状二つ存在しているように見えるという問題です。

同学会には大学当局の公認を得た執行部が存在します。しかし、長らく同学会の活動はほとんど行われておらず、執行部の会議も公に行われていませんでした。しかし、大学運営の矛盾が山積し学生自治の復権が熱望された2012年、有志が執行委員会の再建に着手します。公認の執行部とも連絡して全学で選挙を行い、3000以上の票を得て新たに執行部を建設しました。しかし、その直後に公認の執行部とは連絡が取れなくなったうえ、選挙を認めないという声明が発されました。当局からも選挙を認めないという告示が出ました。しかし、当局の公認は得ていないものの、選挙の結果誕生した新執行部は同学会運動を続けました。その結果生まれたのが「二つの執行部問題」です。

同学会の最高議決機関である代議員会の開催にあたっては、二つの執行部両方に参加の呼び掛けを行っています。

\subsection{6.学外の問題にも取り組んでるの?}
同学会は、学内のみならず、社会全体をも含めた視座に立って運動してきました。それは、京大で起こっている問題が往々にして社会全体の問題の投射になっているからです。例えば、反戦運動があります。学生自治を語る上では、戦争のことに触れざるをえません。第二次世界大戦時、大学は戦争に協力しました。京大も核研究の先頭を走り、731部隊に加担し、学徒動員に行きつき・・・。そんなことになってしまったのは大学が権力のいいなりになってしまったからです。学問がなければ国家は戦争を実行できません。反対に学問を思うままにできたら戦争は容易です。それくらい戦争と学問は結びついています。だからこそ、戦後の大学は大学の自治を追求しました。そして、まさに学問を実践する学生こそがその主体となるべきです。

しかし、国家権力はいつの時代にも学生の自治を破壊しようとして攻撃を仕掛けてきました。そして今、世界は戦争の危機に直面しています。既にウクライナでは戦争が始まっています。東アジアの情勢も急激に悪化する中で、日本においても急激な軍拡が進められています。その中で、学生の自治を破壊する攻撃はこれまでにないほどの激しさで襲いかかっています。

戦争の問題は京大だけで取り組める規模の問題ではありません。だから、歴史的にも他の大学の学生や、学生以外とも連帯して反戦運動を闘ってきました。その要請は今も変わりません。今後も様々な分断を乗り越えて、学内、学外の諸団体とも協力していきます。

\subsection{7.なんで熊野寮のパンフレットに同学会の記事が載ってるの?}
\emphbf{熊野寮自治会が「全学自治会建設」を掲げているからです!}

上にも書いたように、規約に則った同学会の運営を目指していますが、規約に則っているだけでは意味がありません。学内の全ての現場で、隅々まで学生自治を行き渡らせて、京都大学に学生権力を樹立することを目標にしています。そんな中身の詰まった学生自治を実現して、学内の学生自治を発展させることに常に注力してきたのが熊野寮です。今年の熊野寮祭では「総長室突入」という企画が行われ\index{くまのりょうさい@熊野寮祭!きかく@---企画}、全学的な問題10項目について熊野寮から総長への申し入れが実力で貫徹されました。

学生自治の発展は、自治寮の発展と利害をともにします。そんな熊野寮や、他の学生自治勢力とも連帯して、運動を発展させていきます!

\subsection{8.終わりに}
同じ京大の学生でも、学部や年齢、国籍\index{こくせき@国籍}、性別、趣味、好きな食べ物・・・そんな違いが時に我々を分断します。でも、京都大学の学生であるという立場を体現して生きていく以上、一つの結節点があります。それが同学会です。全ての分断を乗り越えて、同学会運動の実践の中で巨大な団結\index{だんけつ@団結}を志向すれば、きっとうまくいきます。まずは、集会を見にきたり、会議を聞きに来たりするだけでも良いので、同学会の運動と関わってみて下さい。ここまで読んでくれたあなたが次の学生自治の担い手です。\emphbf{一緒に明るく、強い、途方もない運動を作り上げて、京大から世界を変えていきましょう!}

\subsecdefault
%\subsectionのデザインを元に