\section{ガサにまつわる大学と寮自治会の立場}
\label{sec:gasa_jiti}
	熊野寮には昔からよくガサ(家宅捜索)が入ります. このガサについて説明します.\index{がさ@ガサ}

		\subsection{捜索の原因}
		\emphbf{捜索の原因は中核派だとされているが, その捜索自体が不当であり, 熊野寮が中核派を追い出すことはない. }\footnote{熊野寮自治会「Campus Life News Vol.12 について」(2017年5月16日)参照}\index{ちゅうかくは@中核派}

 		川添信介副学長は2017年2月14日発行の「Campus Life News Vol.12」において, 「熊野寮の捜索」という文章を発表しました. この文章では, 捜索が行われる原因は中核派が住んでいることであり, 追い出すべきであると述べているように見えます. しかし, 熊野寮自治会が中核派を追い出すことはありませんし, いかなる思想や信条を理由に入寮を拒むことは決してありません.

		なぜなら, 家宅捜索が行われる目的が事件の捜査だとは思えないからです. 当時の捜索は「ある中核派の学生が裁判所で退廷命令を下され, 両脇を抱えられながら法廷警備員の右膝を蹴った」という公務執行妨害事件についての捜索でしたが, 事件から10ヶ月以上も経ったあとにその学生が逮捕され, 熊野寮への捜索が行われました.

		「蹴った」という行為の証拠が10か月後の熊野寮から出てくるとは到底思えません. また, 家宅捜索では大量の機動隊員が動員されました. 過去に寮生が捜索を妨害したことはありませんし, 必要のない捜索を, 必要のない機動隊の動員\index{きどうたい@機動隊}とともに行っている理由は, 寮生や近隣住民を怖がらせたり威圧したりするためだというのが寮自治会の見解です.

 		大学の教員にも, 真理を探究する一研究者として, 法的正当性とは必ずしも合致しない「正しさ」について自ら考え行動してほしいものです.

		\subsection{寮生とともに大学当局も抗議する}
		先述の「確約書」項目Eにおいて大学職員も「その場で抗議する」ことになっています. 責任者間だけではなく, 各職員が各捜査員に対して, 現場で即座に抗議する, ということが当局と自治会の間で確認されています.

		過去には, 2009年5月18日付で, 大学職員が立会を行なうために十分な機会と時間を与えずに警察官が熊野寮の敷地及び建物に入ったこと, 玄関を必要もないのに過剰な人数の機動隊員で占拠し, 寮関係者の出入りを規制したことに抗議する文章を当時の副学長が警視庁警視総監と京都府警・大阪府警に申し入れています.

 		しかし, 「玄関の過剰警備に抗議して下さい」という現場での寮生の要請に対して「後日責任者間で行うから, 今はしない」というような返答をする事務職員も確認されており, これは確約違反となります.

 		もちろん, 過剰警備だけでなく無関係な物件の押収・撮影など多くの不当行為に対し, 学生の生活を守る立場として真摯に抗議する職員・教員も少なくないです. しかし, 個人単位で確約の内容を無視する者がいることも事実であり, そういった事実が確認される度, 当局がその点を改善するよう, 自治会から要求をしています.

       