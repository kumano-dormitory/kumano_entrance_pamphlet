\section{ハラスメント加害者にならないために}\index{はらすめんと@ハラスメント|textbf}
\label{sec:harassment}
  \bunsekisha{文責}{人権擁護部\footnote{人権擁護部では, すべての寮生が不快な思いをせず生活できるよう, 様々な取り組みを行っています.}}

  \subsection{人の体にさわらない. }
  「同性同士のボディタッチは友情の証」, 「異性へのさりげないボディタッチはモテるコツ」などと思っていませんか?その行為は人を不快にさせるには十分です. やめましょう.
  
  \subsection{公共スペースでは下ネタ・猥談はしない. }
  下ネタ・猥談は不快になる人がいます. 過度のいじりや不謹慎ネタなども同様です. その場のノリや勢いだけで発言しないように気を付けましょう. \index{しもねたをしない@下ネタ(をしない)}
  
  \subsection{人の容姿, 私生活を評価することを言わない. }
  人の容姿や私生活はその人のもの, 他の人がとやかく言ったり評価したりすることではありません. 
  
  \subsection{「女性/男性はこうである」など性別によって人のあり方を決めつけない.}
  「女の子なのに化粧しないの? 」とか「男性が奢らなきゃ」とか思ったことありませんか? 性別による一方的な決めつけに苦しめられている人がいます. 性別にかかわらず自分の生き方は自分で決めるものであって, 他の人が押し付けるものではありません. 
  
  \subsection{怒られない雰囲気に甘えない. }
  相手や周囲がはっきりとNGサインを出さなければOKではありません. 嫌な思いをしていても, その場の雰囲気に合わせてニコニコしているだけかもしれません. 
  
  \subsection{大学生には彼氏/彼女がいて当たり前と思わない. }\index{かれし@彼氏}\index{かのじょ@彼女}\index{せくはら@セクハラ}
  「恋人欲しい!」と思って気になる人にグイグイいくと, 相手はけっこう怖い思いをしているかも. 寮は生活の場であって出会いの場ではありません.
  
  \subsection{もし「それ問題だよ」と指摘されたら}
  「意図」で反論しないようにしましょう. 相手を傷つけてしまったり不快にさせてしまったりした場合, あなたがどのような意図でそれを行ったかは全く関係ありません. 自分の行動の何を問題だと指摘されているのか, まずは丁寧に聞きましょう. またそれと同時に, 周囲の人が加害者の意図を取沙汰して責めないようにすることも忘れてはならないことです. 

  