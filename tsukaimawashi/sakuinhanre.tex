
  {\large 前ページの索引凡例}
  
  
  \begin{figure}[H]
    \begin{minipage}[t]{.48\textwidth}
      \small
      \komoku{見出し語}\\
      パンフレットの中から、作成者の独断と偏見にもとづき抽出した。必ずしも本文の語と一致しないものも存在する。(例:「寮食は寝てると食べれない」を「寝坊」として索引している)

      \komoku{詳述頁}\\
      その項目について詳述されている頁を太字にしている。以下の例では、58頁が最もコンパについて詳しく書いてある頁である。
      \begin{quote}
        コンパ\dotfill 7, 16, 52, \textbf{58}, 66, 95
      \end{quote}

      \komoku{副見出し}\\
      合成語については以下の例のようにした。()のように「---」の部分を補って読んでほしい。
      \begin{quote}
        自治\\
        \qquad ---とは何か(「自治とは何か」)\\
        \qquad ---の基本精神(「自治の基本精神」)\\
        \qquad ---のよさ(「自治のよさ」)\\
      \end{quote}
    \end{minipage}
    \hfill
    \begin{minipage}[t]{.48\textwidth}
      \small
      \komoku{「を見よ」参照}\\
      以下の例の場合「風呂」について知りたければ「シャワー」の項を見てほしい。(「風呂」については「シャワー」\kenten{を見よ})
      \begin{quote}
        風呂\dotfill シャワー
      \end{quote}
    
      \komoku{「をも見よ」参照}\\
      以下の例の場合「ハラスメント」について知りたい場合、6, 10頁だけでなく、「セクハラ」「アルハラ」「アカハラ」の項も参考にせよ。(「ハラスメント」については「セクハラ」\kenten{をも見よ})
      \begin{quote}
        ハラスメント\dotfill 6, 10,\\
        \qquad $\longrightarrow$ セクハラ, アルハラ, アカハラ
      \end{quote}
    \end{minipage}
  \end{figure}
  

\bunsekisha{索引作成}{銭苔、銅鑼みがき、ハニホコ、仏子、雀かす}
