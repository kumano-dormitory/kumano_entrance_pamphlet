\setcounter{\thechapter}{-1}
%↓を「第0章」と表示するため
\chapter{この冊子を読むための道案内}
\bunsekisha{文責}{きなこまみれ}

この冊子は『熊野寮入寮パンフレット2023』です。僕らが住んでいる京都大学の学生寄宿舎「\singo{熊野寮}」の紹介を行っています。ですが見ての通り少々分厚いので、編集者の僕が道案内をしていきたいと思います。

「下宿先決まっているから」\index{げしゅく@下宿!さがし@---探し}と冊子を閉じようとしているそこのあなた!読んでからでも遅くはありません。面白い記事もいっぱいあって、暇つぶしにもなるので、ぜひ息抜きにでも。

\subsection{熊野寮とはなんぞや?}
熊野寮は京都大学に学籍を持つ者なら誰でも住めます。\emphbf{男女、院生・学部生、正規生・非正規生、留学生・日本人関係なく}住めます。しかも維持費(寮費)はたったの\textbf{4,300円}。水光熱費や自治会費込みでこの値段です。安いですね。興味出てきた?設備は「熊野寮概要」(p.\pageref{sec:abst})、詳しい入寮方法は「募集要項」(p.\pageref{admission})に書いてあります。

\subsection{熊野寮での生活}
熊野寮は2〜6人部屋が普通で、1人部屋はありません。その他、炊事場やシャワー室、トイレを始め共用のものがたくさんあります。そんな熊野寮での生活を知りたい方は午後三時さんの「くまのせいかつとその他もろもろ」(p.\pageref{sec:kumanoseikatu})がお薦め。シャワーやトイレ、部屋の物品について丁寧に綴られています。