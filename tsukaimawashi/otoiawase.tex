\section{お問い合わせ・ご相談}\label{sec:otoiawase}
熊野寮では、寮内外からの問い合わせ・相談に対して複数の窓口を設けています。ここでは、入寮を希望するみなさんに向けて、熊野寮の問い合わせ先・相談体制についてご紹介します。\index{くまのりょう@熊野寮!のれんらくさき@---の連絡先}

\subsection{熊野寮自治会}
\begin{itemize}
\item メールアドレス:\url{info@kumano-ryo.com}
\item 電話番号:075--751--4050、075--751--4051
\item 受付時間:8:00~23:00
\end{itemize}
\noindent ※熊野寮は、業務が苦手であったり時間がかかったりする寮生含め、皆が当番\index{じむしつ@事務室!とうばん@---当番}に入り運営される自治寮です。そのため、電話の応対に際しご不便をおかけすることがあります。ご理解のほど宜しくお願い致します。

\subsection{入寮に関するお問い合わせ}
\noindent 入寮選考担当のメールアドレス:\url{interview@kumano-ryo.com}

返信まで3日から4日程度のお時間を頂いております。お急ぎの用件に関しましては、熊野寮自治会の項目に記載の番号まで、お電話にてお問い合わせください。直接来ていただいて、寮内を見学することもできます。寮生が案内しますので、入り口右側の事務室にお声かけください。\index{にゅうりょう@入寮!にかんするといあわせ@---に関する問い合わせ} 

\subsection{人権擁護部相談メール}
\noindent 人権擁護部のメールアドレス:\url{kumano.jinken@gmail.com}

人権擁護部では、すべての寮生が不快な思いをせず生活できるよう、様々な取り組みを行っています。入寮後、寮内で生じたトラブル\index{とらぶる@トラブル}や生活上の悩みなどがあれば上記のメールアドレスへお気軽にご相談ください。

また、見学・入寮選考などで来寮の際にハラスメント被害を受けた/見聞きした場合や、入寮にあたってプライバシー\index{ぷらいばしー@プライバシー}に関わるご相談がある場合などには、相談メールまでご連絡ください。

\subsection{女子寮生向けハラスメント相談窓口}\index{じょしりょうせい@女子寮生!むけはらすめんとそうだんまどぐち@---向けハラスメント相談窓口}
\kkomoku{当相談窓口について}

「この人には絶対に相談したことを知られたくない!」「異性相手に絶対に知られたくない悩みがある」「そもそも相談してもいいのかわからない」\index{はらすめんと@ハラスメント|seealsopage{セクハラ, アルハラ, アカハラ}}

寮では(大変残念なことではありますが)このような気持ちを抱える事態が発生することがあります。この相談窓口は、“女子寮生向け“と銘打ってありますが、女子トイレと女子シャワー室を使う寮生に向けて作られたものです\index{じょしりょうせい@女子寮生}。相談員は全員が女子寮生です。セクハラ\index{せくはら@セクハラ}、アルハラ\index{あるはら@アルハラ}、その他トラブルなど、もし将来熊野寮に住んでいて困ったことがあったら、この相談窓口があることを思い出してください。

\kkomoku{相談窓口の使い方}

熊野寮の女子トイレ\index{といれ@トイレ}、女子シャワー室\index{しゃわー@シャワー}、そしてC棟のオールジェンダートイレには、この相談窓口の相談員のLINEのQRコードが貼りだしてあります。自分が相談しやすいと感じる人を選んで、相談のLINEを送ってください。明確に「ハラスメントだ!」と思うようなことでなくても大丈夫です。まずは相談してみてください。また、「この人に知られたらどうしよう」という心配も無用です。相談した内容を誰に共有するか、誰に対応してもらうかは、事態が解決するまで相談した人に決定権があります。
