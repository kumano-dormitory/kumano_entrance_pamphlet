\setcounter{secnumdepth}{3}%\subsubsectionのタイトルを表示するときは3にする。

%パッケージ

\usepackage{xcolor}
\usepackage[dvipdfmx]{graphicx}
\usepackage{picture}%図とか
\usepackage{ulem}
\usepackage[labelformat=empty]{subcaption}
\usepackage{caption}%キャプションの番号消し
\usepackage{multirow}%複雑な表のために
%--------------------------------------------------
\usepackage{xparse}
\usepackage[utf8]{inputenc}
\usepackage{otf}%utfの特殊文字を使うのに使用
\usepackage[top=30truemm, bottom=25truemm, left=20truemm, right=20truemm]{geometry}

\usepackage[stable]{footmisc}%章節などの見出しの中で注を使う
\usepackage{uline--}%行分割可能な下線、ハイライト。ただし、TeXLive等に入ってはいないので、uline--.styを作業ディレクトリに入れておくこと。

\usepackage{epigraph}
\setlength{\epigraphwidth}{.6\textwidth}
%エピグラフを表示するため。タバコの話で使用。

\usepackage{pdfpages}%pdfをそのまま表示するのに使用
\usepackage{subfiles}%texファイルのpreambleを除いて引用する

\usepackage{amsmath} %%数式。多分使わん。
\usepackage{amssymb} %%数式で出てくる特殊記号を使いたい人がいるかも

\usepackage{url} %%URLを貼る。texでは$%&@とかの特殊記号を書くのが面倒なので
\usepackage{float}
\usepackage{wrapfig} %%この2つは図の配置のため
%\usepackage{picinpar}%%図の配置。figwindow環境
\usepackage{floatflt}
\usepackage{pxrubrica} %%ルビが打てます。


\usepackage{multicol} %%複数列の文章
\setlength{\columnseprule}{0.4pt}%複数列の間に線を引く

\usepackage{afterpage}

%c34
\interfootnotelinepenalty=10000%注釈の無き別れを防ぐ
\usepackage{color}           
\usepackage{okumacro}
\usepackage{ulem}

\usepackage{fancybox}%itembox, shadebox, あと何かしらの箱
\usepackage{tcolorbox}%色付きの箱. tcolorboxじゃないとページをまたげないことに注意。
	\tcbuselibrary{breakable} %これがないとtcolorboxがページをまたげないので
  \tcbuselibrary{skins}
\usepackage{ascmac}




%%%%%%%%%%%%%%%%2段組の中で図を使うため
\makeatletter
\newenvironment{tablehere}
  {\def\@captype{table}}
  {}

\newenvironment{figurehere}
  {\def\@captype{figure}}
  {}
\makeatother
%%%%% \begin{figure}[h]の代りに、\begin{figurehere}とせよ


%%%%%%%%%%%%%%%%%%%章の体裁を変更する。
 \usepackage[explicit]{titlesec}%題名や章, 節の見出しの体裁を変える
\titleformat{\chapter}[display]{\bfseries\gtfamily}{}{10pt}{%
  \begin{tcolorbox}[%
    enhanced,%
    colback=white,colbacktitle=white,colframe=black,coltitle=black,%
    boxrule=1.5pt,sharp corners,%
    borderline={1.5pt}{3pt}{black},%
    ]%
    \centering 第\thechapter 章\\
    \LARGE #1%
  \end{tcolorbox}%
}
\titlespacing*{\chapter}{1pt}{-20pt}{1pt}[3pt]%章の上の空隙を0にする。

\titleformat{\section}[block]
  {}
  {\Large\bfseries\gtfamily #1}
  {10pt}
  {\titleline[l]{\titlerule*[5pt]{\tiny\textbullet}}\vspace{-20pt}}

\titleformat{\subsection}[block]{}
  {\large\bfseries\gtfamily ○#1}{10pt}{}
  {\vspace{-20pt}}

\titleformat{\subsubsection}[block]{}
  {〜{\bfseries\gtfamily#1}〜}{10pt}{}
  {}


\newcommand{\subsecdefault}{\titleformat{\subsection}[block]{}
  {\large\bfseries\gtfamily ○#1}{10pt}{}
  {\vspace{-20pt}}

\vskip2\baselineskip
%\vskip2\baselineskipしておかないと、次の節タイトルに本文の末尾が重なってしまうというバグがある。}
\newcommand{\subsecnomaru}{\vskip\baselineskip

\titleformat{\subsection}[block]{}
  {\large\bfseries\gtfamily #1}{10pt}{}
  {\vspace{-20pt}}}
%\subsectionの仕様を変更する。\subsecnomaruで◯をなくす。\subsecdefaultで元に戻す。
\newcommand{\sectionnotitle}{\titleformat{\section}[hang]
  {}
  {\tiny\color{white}{#1}}
  {0pt}
  {}

%\sectionのタイトルを表示しないためのもの、女子寮生座談会で扉ページがあるので導入}
\newcommand{\sectiondefault}{\titleformat{\section}[block]
  {}
  {\Large\bfseries\gtfamily #1}
  {10pt}
  {\titleline[l]{\titlerule*[5pt]{\tiny\textbullet}}\vspace{-20pt}}
%\sectionをもとに戻す。}

%%マクロ
%%%%%%%%%%%%%%%%%%%%%%%%%%%%%%%%%%%%%%%%%%%%%%%%%%
\newcommand{\singo}[1]{{\textgt{#1}}}%全体。新語に。
\newcommand{\komoku}[1]{\noindent{\textbf{〈#1〉}}}%募集要項で使用。
\newcommand{\kkomoku}[1]{\vspace{3mm}\noindent{\textbf{《#1》}}}%項目2。女子寮生窓口で使用。
\newcommand{\kkkomoku}[2]{\vspace{1mm}\noindent \textbf{#1 #2}\\}%項目3。C12座談会の部屋紹介で使用。


\newcommand{\tatespace}{\vspace{1cm}}
\newcommand{\emphbf}[1]{\textgt{#1}}%強調として太字を使用している箇所
\newcommand{\bunsekisha}[2]{\vspace{-2mm}\rightline{{\small(#1:#2)}}}%文責について\bunsekisha{文責}{奥山田}で(文責:奥山田)が小さく右寄で表示。「文責」の箇所は「編集」とか「文字起こし」とかにすればいいかな。

%%%%%%%%%%%%%%%座談会関係のマクロ
\newcommand{\jinbutu}[1]{\vspace{2mm}\noindent #1\\}%座談会の人物紹介。C12座談会で使用した形式
\newcommand{\situmon}{\item[---------]}%インタビュー形式のもので、インタビュアーに使う。
\newcommand{\togaki}[1]{\vspace{1mm} \quad #1 \vspace{1mm}}%ト書き
\newcommand{\hang}[1]{\settowidth{\hangindent}{#1} #1}%1回生座談会人物紹介で使用。
\newcommand{\talker}[1]{\noindent{\gtfamily\bfseries #1}:}%座談会で喋っている人を示す命令。想定している書式は
%\talker{人}喋ったこと<改行×2>
%\talker{次の話者}喋ったこと
\newcommand{\talkerb}[1]{{\gtfamily\bfseries #1}:}%行頭以外で使いたい時
\newcommand{\talkpare}[1]{\noindent (#1)}%座談会中の(から始まる段落に使用。カッコの後ろのコメントアウト

%%%%%%%%%%%%%%\zenkakuspace{n}でn個の全角空白を出力する。wrapfigureの位置調整に必要。
\newcount\kaisu
\def\zenkakuspace#1{
  \kaisu = 0 \loop\ifnum\kaisu<#1
   
  \advance\kaisu by1\repeat
}




\newcommand{\hanten}[1]{\uline[background,color=black,width=1zw,position=.38zw]{\color{white}{\textbf{#1}}}}
%文字の白黒反転する。kyoto scienceの記事をに使用。\ctextはsoulパッケージで使えるようになる

\def\sshatai#1{\makebox[2.25zw][l]{\vphantom{#1}\rotatebox{-48.8}{\scalebox{0.875}[1.143]{\rotatebox{41.2}{\smash{\rlap{#1}}}}}}} %和文斜体を行うためのマクロ。改行に対応できない。

\newcounter{footnote1}%これらは、minipageの中で脚注をつけるために(結構めんどうなことをしています。)1day.texにて
\newcounter{footnote2}

%%%%%%%%%索引の設定
\usepackage{makeidx}
%\usepackage[columns=3,font=footnotesize]{idxlayout}%改頁しない
\usepackage[override=false]{seealso}%|seealsopage{aaaa}でをも見よ参照
\makeindex%idxファイルを作れという命令
\seealsosetup[also]{nameformat=#1}
\def\seename{}
\def\alsoname{\\ \quad $\rightarrow$}
% \def\alsosee#1#2{
%   #2\n
%   $\longrightarrow$#1
% }
%「を見よ参照」を二重矢印、「をも見よ参照」を矢印に。
