\section{ベーシストコンパ}
\bunsekisha{文責}{熊野寮ベーシスト}
以下は熊野寮祭2023内で行われた「ベーシストコンパ」において交わされた会話の内容を文章化したものです。ベース経験者の方、大学からベースを始めようと思っている方、ぜひ読んでみてください!\par また、軽音をやってみたいけど何の楽器をしようか迷っている方、ベースは始めやすくて、比較的すぐに上達できます!ベースをやろう!\par 全員とりあえずMUCの会議に出席しよう!
\vskip\baselineskip
\noindent ※発言者の名前はその人の持っているベースの名前になっています。

\begin{multicols}{3}

\vspace{10mm}
\noindent\subsecnomaru{\large 〈ベーシストが好きなバンドは何ですか〉}

\talker{フランソワーズ}フレデリック\footnote{日本のロックバンド。好き。}。エフェクター\footnote{ギターやベースなどの音色を変える機材}がゴリゴリにかけててめっちゃいい。


\talker{ブラックチャンダー}ぽちょぽちょ言ってるやつとかありますよね。

\talker{フランソワーズ}そうぽちょぽちょ言ってるやつとか、リバーブとかかけてて面白い。

\talker{ブラックチャンダー}でもピック弾きで難しいですよね。

\talker{フランソワーズ}一回やったことあるけどめっちゃむずかった。

\talker{ブラックチャンダー}僕ピック弾きできないんですよね。

\talker{フランソワーズ}できないの!?つまりピック弾きできるようになればブラックチャンダーを越えられるということか、、、

\talker{一同}笑

\talker{こん太郎}無難にユニゾン\footnote{UNIZON SQUARE GARDENの略。日本のロックバンド。かっこいい。}の田淵さん。

\talker{フランソワーズ}好きな理由は?

\talker{こん太郎}踊ってるから

\talker{一同}笑

\talker{フランソワーズ}パフォーマーとしてね

\talker{こん太郎}そう、パフォーマーとして

\talker{フランソワーズ}ベース弾きながら踊りたいよね。Y(某ギタリスト)とかすごい

\talker{こん太郎}田淵さんはあれ弾きながら踊れるのかよ、ておもう。

\talker{フランソワーズ}もっと簡単なことしかしてないのにできない泣

\talker{ブラックチャンダー}あんなに踊ってるのにちゃんと単音引きじゃない難しいフレーズですごい。

\talker{こん太郎}いずれ踊れるようになった時に買ったやつがこれ(徐にワイヤレスシールド\footnote{ギターやベースとアンプをつなげるケーブルのこと。}を取り出
す。)

\talker{一同}おー

\talker{フランソワーズ}これなんですか

\talker{こん太郎}これワイヤレスシールドです

\talker{ブラックチャンダー}すげー

\talker{フランソワーズ}K(某ギタリスト)さんとかが使ってるやつね

\talker{こん太郎}これで俺はいつでも踊れる。技術さえあれば、、、

\talker{一同}(ワイヤレスシールドを観察)

\talker{フランソワーズ}もともと吹奏楽部でベースをやっていたが、コードついたまんま動き回ってた。つまり気合いの問題!

\talker{ブラックチャンダー}吹奏楽って動き回るんですか、、?

\talker{フランソワーズ}足あげたり、こういうのしたり、、

\talker{こん太郎}これでサークルモッシュ\footnote{観客が円を描きながら走り回る行為のこと。}参加したいんですよ

\talker{フランソワーズ}じゃあ絶対コード邪魔やな。でもベースが壊れそう、、

\talker{ブラックチャンダー}確かに、ペグ\footnote{チューニングを合わせるための器具。
}とか取れそう

\talker{フランソワーズ}ベース長いし。

\talker{ブラックチャンダー}人の頭当たりそう

\talker{フランソワーズ}殺人者になっちゃう

\talker{こん太郎}それ考えてなかった、

\talker{一同}笑
 

\talker{ブラックチャンダー}suchmos\footnote{日本のロックバンド。ソウルとかジャズとかHIPHOPとかいろいろ影響受けてて良い}とかですかね

\talker{フランソワーズ}いいね。好きポイントは

\talker{ブラックチャンダー}結構細かい、、(?)弾いてて気持ちいいですね。

\talker{フランソワーズ}やったことあるの?

\talker{ブラックチャンダー}ありますね。あとホルモンとかも好きですね。ゴリゴリにやる感じ

\talker{フランソワーズ}動いてる感じが好き。あと、圧がある感じのベースも好き。例えば須田景凪\footnote{バルーンの中の人。}とか、シンプルなベースだけどめちゃくちゃ抜けてくる。

\talker{ブラックチャンダー}めっちゃ聞こえますよね。際立ってる。

\talker{フランソワーズ}須田景凪はベース好きなんだと思う。

\talker{フランソワーズ}ベースラインってシンプルがいい?動きたい?

\talker{ブラックチャンダー}動いた方が楽しい。

\talker{こん太郎}動きたい。

\talker{ブラックチャンダー}練習する時はそんな動くなよって思いますけどね笑

\talker{フランソワーズ}それはめっちゃ思う

\talker{ブラックチャンダー}でもそれを練習してちゃんとできるようになって、本番やると、あー楽しいなってなる。

\talker{フランソワーズ}でもどんなに頑張ってもベース聞こえないじゃん。それに気づいちゃった。

\talker{ブラックチャンダー}それ考えたら終わりですね笑

\talker{こん太郎}聞こえないけどミスったら聞こえる。

\talker{ブラックチャンダー}コードから外れたら聞こえますよね。

\talker{こん太郎}そう

\talker{フランソワーズ}腹立たしいよね笑

\talker{一同}笑

\talker{こん太郎}他の人は?

\talker{R}Nothig’s \! Carved \! In \! Stone\footnote{日本のロックバンド。かっこいい。}ですね

\talker{一同}知らない、、、

\talker{Y(某ドラマー)}ストレイテナー\footnote{日本のロックバンド。かっこいい。}のベーシストとおんなじ人。

\talker{一同}へー

\talker{フランソワーズ}どんなところが好きですか?

\talker{Y}チャンレンジングな感じ。ガンガンベースかけて、曲の象徴になるフレーズを自分で弾いたりする。そこスラップ入れるんや、みたいな。

\talker{ブラックチャンダー}きいてみよ

\talker{フランソワーズ}なんていう人ですか?

\talker{Y}ひなっち

\talker{フランソワーズ}調べてみよ。

\talker{ブラックチャンダー}ZAZEN BOYS\footnote{日本のロックバンド。好き。} もこの人なんですね。

\talker{フランソワーズ}マニアが出てきた笑

\talker{ブラックチャンダー}他なんかありますか。

\talker{R}the who\footnote{イギリスのロックバンド。イギリス三大ロックバンドの一つ。ジョン・エントウィッスルのベースめっちゃかっこいい。モンゴルにも同じ読み方のThe Huというバンドがあるので混同しないように。}ですかね

\talker{一同}おー

\talker{ブラックチャンダー}七福は?

\talker{七福}ラルク

\talker{ブラックチャンダー}いいね

\talker{ゆきプロ}ベーシストコンパここですか

\talker{ブラックチャンダー}あ、ジャズ研の、、

\talker{ゆきプロ}そうそう、新歓だけ会った笑ウッドベース\footnote{主にジャズで使用されるベース。
}もいけますか?

\talker{一同}いけます

\talker{ゆきプロ}ロシア語もいるよね

\talker{ブラックチャンダー}いる、、今日は行ってないけど笑。ベーシストが好きなバンドある?

\talker{ゆきプロ}King Gnu\footnote{日本のロックバンド。かっこいい。}

\talker{一同}あー

\talker{ブラックチャンダー}King Gnuいいよね。むずいけど

\talker{フランソワーズ}弾けるとかっこいいよね

\talker{ブラックチャンダー}白日とか良い

\talker{ゆきプロ}飛行艇とか学祭のバンドでやった

\talker{フランソワーズ}飛行艇が1番好き

\talker{ブラックチャンダー}フィルター\footnote{エフェクターのジャンルの一種。}みたいなのかかってるよね。ファズ\footnote{エフェクターの一種。}とかも。

\talker{ゆきプロ}そうそう。あとサスフォー\footnote{Suspended 4thの略。名古屋のロックバンド。かっこいい。ベースむずい。}とかもいい

\talker{一同}あー

\talker{ブラックチャンダー}いいね

\talker{ゆきプロ}サスフォーはやばい

\talker{フランソワーズ}サスフォーは根強い人気がありますこの寮では

\talker{ブラックチャンダー}サスフォーは名古屋で路上ライブやってるから、高校の帰りとかたまたま遭遇したりしてた。

\talker{一同}うらやま

\talker{フランソワーズ}難しいよね。セッションみたいな感じで、コピーするのがむずい
 

\vspace{10mm}
\noindent\subsecnomaru{\large 〈ベーシストを始めたきっかけは何ですか〉}

\talker{フランソワーズ}高校から吹部でコントラバスやってて、そのまま低音の虜になり、

\talker{放送}(只今から嘘議案コンテストを食堂で開催します。(某2022寮祭副実、某マイ将皇帝)、(某企画責)は強制退寮嘘議案が出ているので必ず出席してください。)

\talker{一同}(爆笑)席移動しますか、、
 

\talker{フランソワーズ}始めたきっかけ。コントラバスは人が少ないのでやってた。大学入ってバンドでもやるか、てなってエレキベースを始めた。

\talker{ゆきプロ}大体同じ。でも大学入って、オーケストラ入ってる。

\talker{フランソワーズ}コントラバスたまに弾きたくなる。

\talker{ゆきプロ}中高のコントラバスのベースラインはしょうもないやつしかないよね。

\talker{フランソワーズ}そう。ロングトーンで一曲終わる笑

\talker{ブラックチャンダー}弓で弾くんですか?

\talker{フランソワーズ}そう、弓とたまにピッツィカートっていう指で弾く奏法。でも大抵聞こえない笑

\talker{ゆきプロ}チューバに全部取られる。いる意味ない笑。でもオーケストラ入って、難易度が50倍くらいになった。

\talker{フランソワーズ}譜面も圧倒的に難しくなるよね、コントラバスでもめちゃめちゃ動く。

\talker{ブラックチャンダー}僕はそんな高尚な感じではないんですけど、中二くらいの時にBiSHっていうアイドルを推してて、そのメンバーのアユニ・Dっていう子が好きだったんですけど、その子がベース始めて、PEDRO\footnote{アユニ・Dによるソロプロジェクト。よい。}っていうバンドのベースボーカルでやっててそれで始めた。それで始めた。中三くらいから。親に今のベースをかってもらった。

\talker{ゆきプロ}長いな

\talker{フランソワーズ}5年の付き合い?

\talker{ブラックチャンダー}そうですね。最初中古で買って、

\talker{フランソワーズ}いいなー。中学時代にそういう電子系の楽器に出会いたかった。

\talker{ゆきプロ}フェンダー\footnote{楽器のメーカー}の何?

\talker{ブラックチャンダー}エアロダイン2っていうジャズベ\footnote{ジャズベースの略。ジャズベとプレベの違いは割愛。}。多分日本製だった気がする。中古で買ったけど、最初かなり弦高が低かった。

\talker{ゆきプロ}ピックアップが普通と違うね。プレベ\footnote{プレシジョンベースの略。}みたいなピックアップがついてる。

\talker{ブラックチャンダー}そう。プレベの感じで(?)形はジャズベ。(?)

\talker{ゆきプロ}いいなー。なんかすごいものを拾ってるんじゃない?笑

\talker{ブラックチャンダー}あと、弦高が低かったおかげで、スラップ\footnote{親指を叩きつけて鳴らす弾き方。楽しい。}がとてもやりやすかった。弦高が高いとプル\footnote{指で弾く弾き方。楽しい。}しづらい。
 

\vspace{10mm}
\noindent\subsecnomaru{\large 〈好きなベースライン〉}

\talker{フランソワーズ}King Gnuの飛行艇とかいい。

\talker{ブラックチャンダー}オドループとかもいい。テレテレテレテテのところ。

\talker{ゆきプロ}めっちゃいい。フレデリックは高校生のころ飽きるほど聴いてた。笑

\talker{ブラックチャンダー}フレデリック聴いてくとよくわからん歌とかあるよね。死んだ魚の目をした魚みたいな笑

\talker{フランソワーズ}フレデリック歌詞に意味ないから

\talker{ゆきプロ}ほんまに適当なんよな。歌詞に意味ない。韻で作ってる感じ。

\talker{フランソワーズ}スパム生活か!

\talker{ブラックチャンダー}そうですそうです。

\talker{ゆきプロ}あったなーそんなの。

\talker{ブラックチャンダー}N好きなベースラインある?

\talker{N(某ギタリスト)}うーーん

\talker{ブラックチャンダー}東京事変\footnote{みんな知ってるっしょ。}とか。

\talker{N}ベースはちょっと眼中になかった。笑

\talker{フランソワーズ}ベース聞けよ。ベースが一番かっこいいだろ。

\talker{ブラックチャンダー}うるうるうるうとかいいよね。

\talker{N}ひいてみて。

\talker{ブラックチャンダー}(ベースを弾く)

\talker{ブラックチャンダー}、、、他なんかありますか?

\talker{フランソワーズ}rayのテレテッテテレテはテンション上がってた。でもあんまり聞こえない。まじでベースだけなのに自分で聞こえなかった。(スマホでrayを流す)

\talker{ブラックチャンダー}生音だから聞こえないのかも
 

\vspace{10mm}
\noindent\subsecnomaru{\large 〈ベースやってて良かったことはありますか〉}

\talker{ゆきプロ}ベースができたこと

\talker{ブラックチャンダー}ベースやってるんですよね。ってドヤれる。

\talker{ゆきプロ}ギターとなんかちゃうんて100回くらい聞かれる。

\talker{ブラックチャンダー}それに関してはもう諦めてる。笑ギターって言われてもギターだよって返してる。

\talker{フランソワーズ}なんかバンドやってるんだよね?ギターだっけ、、、うん。って返す笑。ベースギターともいうし。

\talker{ブラックチャンダー}弦が何本とか普通の人はわからないから。

\talker{こん太郎}あえて五弦って言ってる。

\talker{フランソワーズ}気持ちいいフレーズに出会えるとベースやってて良かったなーってなる。

\talker{ブラックチャンダー}普通の人が知らないようなフレーズありますもんね

\talker{フランソワーズ}私しか知らないこの曲のいいところ、みたいな。

\talker{ゆきプロ}ベーシストって人知れず気持ちよくなれるからいい。

\talker{ブラックチャンダー}飛行艇のこのフレーズとかいいよね。人知れず気持ちよくなってた。

\talker{フランソワーズ}楽しいというより、気持ちいいっていう感覚が強い。ベースならではなのでは?

\talker{ブラックチャンダー}ドラムとかも同じかも知れないですね。

\talker{ゆきプロ}ウッドベースで気持ちよくなることあんまりない。

\talker{ブラックチャンダー}わかる。

\talker{ゆきプロ}もともとあるベースラインを弾くのならいいけど、ジャズだと自分で作るからあんまり気持ちよくならない。

\talker{ブラックチャンダー}自分に気持ちよくなるフレーズを作り出す技量がないんだよね。

\talker{ゆきプロ}そうそう
 

\talker{ブラックチャンダー}あと、曲を注意深く聞くようになった。ベースの音が聞きたいから。

\talker{こん太郎}それによってスマホとかイヤホンの性能がわかるよね。

\talker{ブラックチャンダー}ベースはイヤホンつけないと聞こえないですもんね。

\talker{フランソワーズ}めちゃめちゃ音デカくしちゃう。

\talker{ゆきプロ}スピーカー選ぶ時とかベースの音を基準にする。

\talker{フランソワーズ}いい音楽機材に出会える。

\talker{ブラックチャンダー}機材に厳しくなる。
 


\vspace{10mm}
\noindent\subsecnomaru{\large 〈自分のベースの好きなところは何ですか〉}

\talker{ブラックチャンダー}フランソワーズさんのベース見たことないメーカーですね。

\talker{フランソワーズ}これは寮で拾ったクソ安ベース。私はこれに金を出してません。

\talker{ブラックチャンダー}何て読むんですか

\talker{フランソワーズ}フェルナンデス。ちなみにこの子の名前はフランソワーズって言います。フェルナンデスに似てるからです。笑。あと、二外がフランス語だったのでフランスと組み合わせて、フランソワーズにしました。かわいいですね。色味が好きです。

\talker{ゆきプロ}色がフランスっぽくないけどね。笑

\talker{フランソワーズ}確かに

\talker{ブラックチャンダー}それ何フレット\footnote{指板(弦を指で抑えるための板)に打ち込まれた金属棒のこと。}あるんですか。多くないですか。

\talker{フランソワーズ}24フレットまである。でも、拾ったやつで性能が良くないから、好フレットだとビビって弾けない、、、弦高調整してもどうにもならんかった。

\talker{こん太郎}調整はまだできそう

\talker{フランソワーズ}ネックの反りを直そうとしたんだけど、ここがもうダメになっててどうにもならん。もうそろそろバイバイかも。

\talker{ブラックチャンダー}ベースに名前をつけましょう。

\talker{フランソワーズ}ミンティアさんのベースはミンティアってつけてた。ミント色だから。

\talker{ブラックチャンダー}スティングレイ\footnote{ニュージックマンが出してるベース。高い。}のやつね。

\talker{フランソワーズ}みんちゃんと呼んでいる。

\talker{ブラックチャンダー}こん太郎さんのベース綺麗ですよね。

\talker{フランソワーズ}紺色がいい。

\talker{ブラックチャンダー}指板の色が好きだ。

\talker{ゆきプロ}メープルですか?

\talker{こん太郎}そう。かっこいいよね。絶対にかぶりようがない。誰に言ってもわからないブランドの誰に言ってもわからないベースだから。笑

\talker{ブラックチャンダー}何ていうメーカーですか。

\talker{こん太郎}コンバット。リペアのメーカー何だけど、オーダーで作ってて、かった店がオーダーして作ってもらってたやつ。音もよくって、いろいろ表現できるジャズベ。ひいててとてもたのしい。

\talker{フランソワーズ}お店の人みたいなこと言うな。笑

\talker{こん太郎}まじで楽しい。

\talker{フランソワーズ}こだわりがあるよね。

\talker{こん太郎}コントローラーくるくるするのも楽しい。

\talker{フランソワーズ}五個もある。

\talker{ゆきプロ}何のコントローラー\footnote{ベースについてるクルクル回すやつ。まあ、適当に回していれば自ずとわかってくる。}ですか?

\talker{こん太郎}ボリューム、(なんて言ったかわからなかった)、トレブル、ミドル、ベース\footnote{それぞれ高音域、中音域、低音域のこと。}。プリアンプ\footnote{アンプに信号を送る前に音を増幅させる機材。ベースに入ってたり、エフェクターボードの中にあったり、アンプの上に置いてくれたりもする。}も内蔵してる。アクティブ\footnote{アクティブベースのこと。}、パッシブ\footnote{パッシブベースのこと。アクティブベースとパッシブベースの違いは割愛。簡単にいうと電池いるかいらないか。}も切り替えられる。

\talker{一同}おー

\talker{フランソワーズ}自分の三個のコントローラーも使いこなせてない。笑

\talker{こん太郎}自分も使いこなせてないけど、これだけで音の方向性が出てくる。

\talker{フランソワーズ}可能性が広がるね。

\talker{こん太郎}表現の幅が広がる。楽しい。

\talker{フランソワーズ}何で五弦なんですか。

\talker{こん太郎}低い音出したくないですか?

\talker{フランソワーズ}4弦の形が好き。手が小さいから、こっちの方がいい。

\talker{こん太郎}打ち込みとかの曲になると、どうしても五弦を使うから、できる方がいいなと思った。

\talker{ゆきプロ}五弦の音欲しいなと思ったことはある。

\talker{こん太郎}今のところ五弦使ったのは2回くらい。君の知らない物語とか、ミラーチューンとか。

\talker{ブラックチャンダー}五弦の曲はドロップDにしてずっとやってた。

\talker{ゆきプロ}一番下の音って、チューニング何ですか。

\talker{ブラックチャンダー}B。

\talker{フランソワーズ}ベー。笑

\talker{ゆきプロ}オケ民はベーっていう笑。

\talker{ブラックチャンダー}そうなんや。何語なん

\talker{ゆきプロ}ドイツ語。アー、ベー、ツェー、デー、エー、エフ、ゲー。笑

\talker{ゆきプロ}なんか別の楽器やってたんですか?

\talker{フランソワーズ}フルートやってた。でもクラシックから離れすぎてわからなくなってる。

\talker{こん太郎}このベース、名前はまだつけてない。

\talker{一同}つけましょう

\talker{フランソワーズ}コンバットだから「こん」でいいよ。コンバットでもいい。

\talker{一同}適当すぎる笑

\talker{フランソワーズ}「こんちゃん」で。

\talker{こん太郎}コンバットを名前にしたくない。虫殺してそうだし。

\talker{フランソワーズ}コントラバス(?)

\talker{ゆきプロ}それは嘘。笑

\talker{フランソワーズ}猫に犬ってつけてるようなもんだね。

\talker{ゆきプロ}コンバットの太郎感あるよね(?)

\talker{ブラックチャンダー}こん太郎にする?

\talker{こん太郎}ありかも。

\talker{ゆきプロ}5と紺を合わせて、「ファイビー」とか。

\talker{こん太郎}だったらこん太郎かな。

\talker{フランソワーズ}「こん太郎」決定で。
 

\talker{ブラックチャンダー}七福自分のベースの好きなところある?

\talker{七福}あったかい音がするところ。

\talker{フランソワーズ}そのベースいい値段するやつだよね。

\talker{七福}10万円くらい。一福\footnote{一福サービスという会社の警備員バイト。多くの寮生が働かせてもらっている。}頑張って買いました。

\talker{ブラックチャンダー}一福の注釈書かないと。笑

\talker{こん太郎}ちなみに何福分くらい?

\talker{七福}七福ですね。
 

\talker{N}電波通信のベースかっこいい。

\talker{ブラックチャンダー}その話題終わっちゃった。でもかっこいいよね、電波通信。

\talker{こん太郎}狂乱も好き

\talker{フランソワーズ}いいよね。オーラルのベース良い。

\talker{ゆきプロ}オーラルあるある、狂乱以外知らない。笑

\talker{フランソワーズ}容姿端麗な嘘とかあるやろ。
 

\talker{ブラックチャンダー}僕のベース行きますか。

\talker{フランソワーズ}黒がかっこいい。

\talker{ブラックチャンダー}最初黒買おうと思ってた。

\talker{ゆきプロ}ネックも黒なのいい。

\talker{ブラックチャンダー}ネックも絶対黒が良かった。

\talker{フランソワーズ}指板黒ってあんまりないよね。

\talker{一同}確かに。

\talker{ブラックチャンダー}全部黒にしたかった理由っていうのが、アユニ・Dが持ってたベースっていうのが真っ黒のスティングレイだったから。卵の中まで真っ黒だった。後ろまで全部真っ黒で。

\talker{ブラックチャンダー}あと、弦高。

\talker{フランソワーズ}矯正の末の?

\talker{ブラックチャンダー}矯正してなくて、中古で買って、その時からこのめちゃくちゃ低い弦高だった。なぜか三弦だけちょっと高くなってる。スラップしやすい。

\talker{フランソワーズ}思った。三弦だけ高いよね。

\talker{ブラックチャンダー}このおかげでベース上手くなった。名古屋の楽器屋さんで出会いましたね。

\talker{N}これどうやって音出してるの?

\talker{ブラックチャンダー}プルとスラップていう二つの奏法で音出してる。

\talker{ゆきプロ}ダブルプルができないんだよね。

\talker{ブラックチャンダー}めっちゃ練習した。笑。ロータリー\footnote{スラップ→親指で弾く→プルをすごい速さでやる奏法。説明むずい。楽しい。}も。

\talker{ゆきプロ}名前ある?

\talker{フランソワーズ}オールブラックス

\talker{ブラックチャンダー}バスケのチームみたい笑

\talker{フランソワーズ}りおちゃんにしたら

\talker{こん太郎}ややこしくなる笑。クロちゃんとか

\talker{ゆきプロ}弊害がある。笑

\talker{フランソワーズ}好きな食べ物は?

\talker{七福}ベースの?

\talker{こん太郎}違うだろ笑

\talker{七福}そこまで没入しているのかと思った。笑

\talker{ブラックチャンダー}チャンダー\footnote{熊野寮生御用達の本格インド料理屋。うまい。}ですね。

\talker{フランソワーズ}ブラックチャンダーで。

\talker{一同}かっこいい。笑

\talker{ブラックチャンダー}いいですね。

\talker{ゆきプロ}ブラックサンダーみたいでかっこいい。
 

\talker{ブラックチャンダー}七福のベースにも名前つけますか。

\talker{こん太郎}七福は?

\talker{一同}良い。

\talker{七福}この子は七福くんです。
 

\talker{ゆきプロ}白いフェンダーなんですけど、仮決めで雪だるまくんと呼んでいた。この際確定してしまいたい。なんかありますか。

\talker{ブラックチャンダー}雪だるまプロ\footnote{京大の映画作るサークルだった気がする。よく知らない。}。

\talker{一同}笑

\talker{ゆきプロ}結構いい映画作るよね

\talker{フランソワーズ}ゆきプロにしようよ。

\talker{ゆきプロ}プレベだしちょうどいいかも。ゆきプロ君で決定で。
 

\talker{ブラックチャンダー}あとリッケンバッカー\footnote{高いベース。}も持ってる。

\talker{フランソワーズ}名前は?

\talker{ブラックチャンダー}リッケンバッカーで。笑これも30万した。親に借りて買いました。

\talker{フランソワーズ}こん太郎のベースはどうやってかったの?

\talker{こん太郎}貯蓄と、その月の生活費全てつぎ込んだ。

\talker{フランソワーズ}でもその月死んだんじゃない?

\talker{七福}どうやって貯めたんですか

\talker{こん太郎}浪人期は何もしないんだよね。それでお小遣いが溜まっていく。そもそもお金を使わない性格だからっていうのもある。

\talker{フランソワーズ}浪人だったんだ。

\talker{こん太郎}同いだよ。泣

\talker{こん太郎}その月は3000円で半月過ごしました。全部自炊した。笑
 


\vspace{10mm}
\subsecnomaru{\large 〈エレちゃん到着〉}

\talker{一同}遅い。自分のベースの好きなところは?

\talker{エレちゃん}安くて音がいいところ。五弦だけどちゃんとが弦がなる。

\talker{フランソワーズ}五弦を使ったことは何回くらい?

\talker{エレちゃん}ダウンCチューニングでホルモンとかよく使う。チューニングせず使えて楽。

\talker{フランソワーズ}名前はなんですか

\talker{エレちゃん}名前ないよ

\talker{フランソワーズ}エルレのステッカー貼ってあるし、エレちゃんかな。


\end{multicols}

\vspace{10mm}
\noindent\subsecnomaru{\large 【編集後記】}

GEZANあるある:世界観すごい
The Whoはとりあえず、Quadropheniaっていうアルバムを聴こう。ベーシストには刺さるはず。

こんな変な文章を最後まで読んでくれてありがとうございました。これで君もベーシストの仲間入りだね。(ブラックチャンダー)

