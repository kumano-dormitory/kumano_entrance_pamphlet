\section{ 熊辞雲編纂室だより}
\author{仏子}

\subsection{『熊辞雲』――熊野寮のことばの辞典}

熊野寮時間では世間の午前中が「夜」、昼過ぎが「朝」ということになっている。私は昨年の夏、毎日のように「朝」から『熊辞雲』の編纂に取り組み、ほか数名の寮生・OPとともに第三版を完成させた。この文章では、熊野寮のことばの辞典『熊辞雲』について紹介する。『熊辞雲』の全文は掲載しないので、入寮してから受け取ることを楽しみにしていてほしい。
熊野寮に入寮した当初、寮内で使われていることばで意味がわからなかったものの一つに「団結」がある。「団結」は、熊野寮において最も重要なことばの一つである。以下に、一般的な「団結」と熊野寮での「団結」の違いについて述べる。
まず、アクセントが異なる。『新明解国語辞典』によれば、「団結」のアクセントは、「カレエダ(枯れ枝)」と同じである。それが熊野寮では、「タンポポ」のアクセントで発音される。これは、「連帯」や「貫徹」「方針」「総括」も同様である。

一般的な辞書では、「団結」は以下のように記述されている:
\begin{quote}
人々が力を合わせ、強く結びつくこと。「一致―」(岩波国語辞典)

多くの人が、同じ目的のもとに信頼し合い、しっかりとまとまること。「一致―」「固い―」(旺文社国語辞典)

目的を達成するために、一人ひとりは弱い者同士が団体を作り、強い力を持って行動すること。「―を図る/―を訴える/一致―する/―力・―権・―心・大同―」(新明解国語辞典)
\end{quote}

一方、熊野寮での「団結」は、より具体的な使用例に基づいており、次のような結合語が見られる:
\begin{itemize}
    \item[A] 「寮生の団結」「団結をつくる」「団結を深める」
    \item[B] 「圧倒的団結」「団結破壊」
    \item[C] 「バラライカ奏者と団結した」「外部のサークルと団結」
\end{itemize}

 Aのグループは、一般的な結合語とほぼ一致する。Bも、寮外ではあまり見ないことばの組み合わせだが、そこまで不自然ではないだろう。一方Cの結合語は、意味はわかるものの、寮外の人の目には新鮮に映るのではないだろうか。
 それでは、語の意味を比較してみたい。高校生までに親しんできた「団結」の使い方といえば、クラスマッチや受験に向けての「クラスの団結」かもしれない。あるいは、「労働者の団結権」を思い浮かべる人もいるだろう。いくつかの辞書で「団結」を引いて、以下に並べてみる。

人々が力を合わせ、強く結びつくこと。「一致―」(岩波国語辞典)
多くの人が、同じ目的のもとに信頼し合い、しっかりとまとまること。「一致―」「固い―」(旺文社国語辞典)
目的を達成するために、一人ひとりは弱い者同士が団体を作り、強い力を持って行動すること。「―を図る/―を訴える/一致―する/―力・―権・―心・大同―」(新明解国語辞典)
 
 上の三つの語釈について検討する。『岩波国語』がどちらにも当てはまる最低限の要素を書いているのに対して、『旺文社』と『新明解』は、それぞれ「同じ目的のもとに」「目的を達成するために」と、人が何のために団結するのかまで明示している。さらに『新明解』の語釈では、「一人ひとりは弱い者同士が団体を作り」の部分や「―権・大同―」といった用例によって、「労働者の団結」のような場合の「権利を守り獲得するための団結」が記述されている。
次に、熊野寮における「団結」の意味を考えたい。寮生がなぜ何のために団結するか、実例を取り上げてみよう。
「大学職員や警察による妨害や弾圧、処分や逮捕……一人では立ち向かえない権力の攻撃も、本気で団結して一緒に行動すれば必ず乗り越えられる」(熊野寮HP掲載「総長室突入に関する声明」より)
「時計台前で権力との最前線で大量の学生が団結して声を上げています!クスノキ前には誰にも把握しきれない程の学生が彼等の勇姿そしてこの一揆の勝利を見届けています!権力は包囲された!」(熊野寮広報局Twitterアカウント、2024年12月2日のツイート)
上記二つの実例からは、寮での「団結」が、「国家や大学当局といった権力に対抗する団結」「弾圧に抵抗するための団結」「権利を守り獲得するための闘う団結」であることが読み取れる。そのため、「クラスの団結」のような「団結」の使い方に高校生まで馴染んできた人にとって、寮での「団結」の意味が自分のものになるまでに時間がかかるのだろう。
 さらに、寮ではこの「労働者の団結」の意味の「団結」が、もっとくだけた形で使われることが多い。
D)「乗ったタクシーの運転手が、熊野寮が闘ってることにめちゃくちゃ理解を示してくれた。団結」「外部のサークルと団結している寮生」「バラライカ奏者と団結した」
 以上を踏まえ、いよいよ熊野寮における「団結」の意味を記述してみよう。
 
だん-けつ【団結】(名・自スル)(アクセントはタンポポに同じ)①一人一人では国家権力・大学当局との力関係において不利な存在である寮生や学生が、上位の権力に対して要求を突きつけ交渉を迫るために、自治の理念を共有して一丸となること。激励や奮起を表すために、ことばの結びに使われることがある。「寮生の—」「—をつくる」「—を深める」「—の強化」「寮外への—の拡大」「圧倒的—」「—破壊」「ソビエトの文化理解において圧倒的勝利した。—」「タクシーの運転手が、熊野寮が闘ってることにめちゃくちゃ理解を示してくれた。—」②仲良くなること。仲良くすること。「コンパによる寮内—」「バラライカ奏者と—した」「シーシャ同好会と—している寮生」[関連語]連帯
 
 熊野寮のことばを独特にしているのは、左翼用語である。寮では、「アジる」「自己批判」「決起」「人民」といった用語がしばしば使われる。また、活動家がよく使う「革命的」「粉砕」「ナンセンス!」などのことばもよく聞かれる。建寮60年を迎える熊野寮では、昔からのことばが変わらずに引き継がれる一方で、新たなことばや使い方が日々発生している。そしてその中には、役目を終えて忘れられていくことばもある。熊野寮のことば群は、時々刻々と色やかたちを変える雲のようである。
空を流れる気ままな雲をスケッチするように、熊野寮のことばを記録した辞典がある。それが『熊辞雲(くまじうん)』だ。この辞書には、寮生の発話やLINEの文章、会議の議案、寮内ポスターなどから採集されたことばが載っている。加えて、実際の使用例に基づく生き生きした用例もふんだんに取り入れられている。
この記事では、第三版編纂の流れ、そして作業の過程で考えたことについて述べる。初版・第二版の編纂や、『熊辞雲』の名前の由来については、入寮パンフレット2022に記事を寄稿しているので、そちらを読んでほしい。ここまでが序文である。私はぜんたい、費やした字の量に値するだけのなにかを書けるのだろうか?


\subsection{書き込んでも書き込んでも白いセル――編纂記録2022~2023}


 『熊辞雲』第二版は、コロナ禍の続く2021年2月に刊行された。第二版では、限られた時間で385語にのぼる見出し語の語釈を書くために、先に見出し語がある状態で、その語の用例を資料で検索するという方法を取らざるを得なかった。しかし本来であれば、資料を読むなかで新しいことばや用法、結合語を探すほうが望ましい。そこで第三版編纂にあたっては、資料をイチから読み込んで丹念な用例採集を行うことを目標に掲げた。そのためには何といっても作業時間を確保することが必要になる。これを実現するために、2022年は4月から毎月1回編集作業の機会を設けることにした。
 編纂の一環としてやってみたいことが二つあった。一つは編纂合宿である。大学の長い夏休みに、朝から晩まで語釈の執筆や検討に勤しみ、夜はわいわいレクリエーションを楽しむ。合宿場所はもちろん、標高1367mの高地に立つ空気のきれいな高級避暑地・熊野寮である。
 編纂合宿は、2022年8月24、25日に一泊二日の日程で実現した。合宿には3人の編集委員が参加し、昼間は食堂で編集作業、夜は映画『舟を編む』の鑑賞と、広辞苑を使った遊戯「たほいや」を行った。お金もかからず、寝るときは各自部屋に帰ればいいだけなので、大変手軽だった。ただ、編集作業自体はあまり進まなかった。
 やってみたかったもう一つは、寮で年2回開催されるイベント中に、用例採集の実況ツイートをすることである。『三省堂国語辞典』の編纂者である飯間浩明氏は、例年大晦日の紅白歌合戦を見ながら、歌詞に現れる新語や新用法をTwitter上に随時メモされており、私はそのツイートを追うのを毎年楽しみにしている。どういうものか理解してもらうために飯間氏のツイートを引用すると、
〈日向坂46「君しか勝たん」の「~しか勝たん」は「~が最高だ」などの意味ですが、私が初めて見たのは2019年のことでした。『現代用語の基礎知識』では2021年版(2020年発行)から載っています。こうして「紅白」で歌われると、全国的定着までもう一歩。坂道シリーズの歌は現代語観察では欠かせません。〉(@ IIMA\_Hiroaki, 2021/12/31)
といった具合である。
飯間氏の著書に、『辞書を編む』(光文社新書、2013)がある。辞書編纂者の仕事について書かれたこの本は、高校時代の私を大いに興奮させ、高校の用語辞典を作らせた。この本を読んでいなかったら、『熊辞雲』を作ろうと思い立つこともなかっただろう。
 話を戻そう。用例採集実況も、Twitter上ではなかったが、2021年前半から実施することができた。こちらから発信するだけでなく、暇をしている寮生が耳に留めたことばを教えてくれたりして、大いに盛り上がった。「直談判」「立場性」「ヘゲる」「朝ビラ」など多くの収穫があった。第三版に反映したことばも多い。現在もイベントごとに実施しており、地味に長寿企画となっている。

 第三版の刊行は当初、2023年2月を目指していた。編集作業を開始して約11ヶ月後の2023年2月17日、私が編纂室のグループLINEに投下した進捗報告がこちらである。
〈今日は、「あ〜う」の項目を書き換えることができました〉
そして、グループにいるOPから投げ返されたメッセージがこちらである。
〈?〉
ぐうの音も出ないとはこのことである。およそ1年もの間、私が編集作業と称して行っていたことは一体何だったのか? Googleスプレッドシートに入力した文字数は決して少なくないはずであった。どうしてこうなってしまったのか説明すると、私がちまちま充実させていたのは見出し語ごとの実例欄で、手入れ(第二版収録語の語釈の見直しと書き換え)と新規収録語の語釈の執筆にはあまり手が回っていなかったのだった。作業回数自体も多くなかったとはいえ、語釈を執筆する以前の段階で1年が費やされてしまったわけである。
このように改訂作業は遅々として進まず、第三版刊行の目処も立たない期間だったが、別の方面では進歩もあった。
まず、挿絵を追加したことが挙げられる。寮生のダリさんに依頼して、民青池のオブジェ(後述)、小鉢、残置札、荷物札の精細なイラストを描いていただいた。肉筆の原画は私が預かっているが、額装して寮の目立つところに飾ったほうがいい。ほんと。
また、『熊辞雲』を頒布する機会にも恵まれた。中でも嬉しかったのは、『熊辞雲』がコミケデビューを果たせたことである。2022年冬のコミックマーケット101において、学寮交流会のブースに置いていただいた。コミケで販売した第二版には、上に述べたイラストを第三版に先行して掲載した。
新たな試みがあった一方で、肝心の改訂作業が進まなかったのはなぜだろうか。その原因は、第三版の具体的な方針が定まらなかったことにある。第二版完成直後には、第三版で目指すこととして「ユーザー目線の辞書作り」という目標を掲げていたが、それは長く地道な編纂作業を貫徹するのに十分なモチベーションとはなり得なかった。結局、第三版改訂の再始動に火をつけたのは、失われてゆくものを記録しておかなければという焦りだった。編纂作業の再開については、章を改めて語ることにしよう。

\subsection{コラム1 新入寮生の対義語}


 熊辞雲編纂室で2024年最もホットだった話題といえば、「新入寮生」の対義語は何であるべきかというテーマであった。熊野寮では、新しく寮に入寮した寮生を「新入寮生」と呼んでいる。春期入寮面接・秋期入寮面接いずれを通っても、次の春に再び新しい入寮者が現れるまで、その年の入寮者は「新入寮生」と呼ばれることになる。では、「新入寮生以外の寮生」は何と呼べばいいのだろうか。
 現在、寮では「上回生」という呼び方が一般的である。例えば、「いまから全寮新歓を始めます。上回生は新入寮生を食堂に連れてきてください」「新入寮生は無料。上回生はカンパをください」「新入寮生は上回生と一緒に仕事に入ってください」などと使う。大学のサークルにおいても、「新入生・上回生」という言い方は普通に行われている。では、寮において「上回生」という呼び方の何が問題なのか。
 それは、2回生以上や院生になってから入寮する人も多いという点にある。2回生以上の人々は、寮外では上回生と呼ばれている。しかし、寮における「上回生」は、寮に入ってから半年ないし1年以上が経過した者を指すのである。新しく寮に入った2回生以上の寮生は、「新入寮生・上回生」という寮の用語よりも、「新入生・上回生」という大学の用語のほうに慣れている。そのため、自分が「新入寮生」なのか「上回生」なのか混乱するのである。実際に私は、「上回生と仕事に入ってください」というLINEのメッセージを見て、とっさに自分自身を上回生に分類して返信してしまった学部4回の新入寮生の例を知っている。そしてこれは、単なる用語の取り違いやすさとして片付けることのできない問題を含んでいるのである。
 2回生以上の学年から寮に入寮した人は、寮外では上回生であるにも関わらず寮内では「上回生」としてカウントされない状態で1年を過ごすことになる。さらに、自他ともに「新入寮生=1回生」という図式が強いと、「自分は新入寮生である」という自己認識にまでひびが入っていく。この「上回生としても新入寮生としてもカウントされない」という意識が問題なのである。それは寮から疎外されているという認識を生み、寮へのコミットを妨げてしまう。
 ここまでのところで、「上回生」という呼称が抱える問題はおわかりいただけただろう。それでは、「上回生」の代わりになる表現Xとして何がふさわしいのか。編纂室の四方山ついでの議論を踏まえ、私の考える表現Xの条件をまとめると、以下の通りになる。
 
①上下関係を含まないこと。
②似た概念の既存の用語を利用するか、そうでなければ全く新しく創り出された用語であること。
③「新入寮生」と対称的なつくりの用語であること。
 
 条件①は、熊野寮において重要な「公平・平等の原則」を踏まえている。熊野寮は、敬語禁止の吉田寮ほどではないが、やはり「すべての寮生は平等である」という原則を大切にしている。条件①を適用すると、例えば「先輩寮生」のような呼び方は望ましくないということになる。
 次に、条件②について説明しよう。寮で通用している「新入寮生」「在寮生」「卒寮生」という用語は、それぞれ「新入生」「在校生」「卒業生」と近似した概念であるがゆえに、意味がわかりやすく浸透しやすい。しかし、一般的な語彙「新入生・上回生」に対応する寮用語「新入寮生・上回生」は、字面が酷似する一方で、前者の組み合わせが指す領域と後者の組み合わせが指す領域がずれているために、すでに述べたような問題を引き起こしてしまった。つまり、Xを過不足なく表現できる既存の用語がないのであれば、意味領域のずれによる混乱を避けるために、手垢のついていないまっさらな新規語を生み出す必要があるということである。
 「先住民」「先住寮生」という候補も浮上したが、これも却下される。「先住民の残していった荷物」などのように、「以前その部屋に住んでいて今は退寮したOP」という意味ですでに使われているからである。「古参寮生」も挙がったが、「厄介な古参」という好ましくないニュアンスを含みそうなため却下された。
 条件③は、ことばの洗練度の問題である。「『新入寮生』の対義語は、字面の上でも綺麗な対称を成していなければ認めない!」という思想である。例を挙げよう。寮で開発された荷物管理アプリ上に、(荷物を)「受け取る」「引き渡す」という一対の操作ボタンがある。「受け取る」に対置することばは、意味を考えれば「渡す」の2字でもいいはずである。しかし、「引き渡す」にすることで、視覚上「漢字+送り仮名+漢字+送り仮名」の組み合わせの4字が揃って美しい。
 『熊辞雲』は、寮のことばのあとを追いかけて記録する辞書である。よって、私が考えたXにせよ、ほかの編纂室員のそれにせよ、寮で実際には使われていない状況を無視し、先行して項目に立てることはない。だからどうか、私の案はあくまで一個人の主張として受け止めていただきたい。
 それではいよいよ、私が寮に定着させたい表現Xを発表しよう。それは「順応寮生」である! 
 上の三条件に適合するかどうか見てみよう。まず①は満たしている。また、「順応」がわかりやすい概念である上に、「順応○○」という語の組み合わせは新しいので、既存のことばと意味領域がかぶって混乱する事態も避けることができる。よって、②もクリアしていると見なすことができる。③についてはその対称性の美しさに、我ながら嘆息するばかりである。「新入寮生」と「順応寮生」はどちらも漢字4字である。のみならず、耳で聞いたときにも「しんにゅうりょうせい」「じゅんのうりょうせい」で2音目の「ん」と4音目の「う」が綺麗に一致しているのである。
 わが用語は現在のところ普及していないが、要は年度末までに定着していればいいのだ。この文章を読んでいる人が入寮するまでに、必ずや「順応寮生」を浸透させてみせる!


\subsection{誰がために改訂は成る――編纂記録2024}


 熊野寮は毎年4分の1ほどの寮生が入れ替わる。入れ替わりが激しいため、一時期盛んに議論されていたことが1、2年後には忘れ去られていたり、ずっと昔からの伝統だと思われているものがわずか数年前に始まったことだったりする。熊野寮は、たくさんの寮生によって絶えず更新され続けている。この現在の状況をどうやって記録に残せばいいのだろうか。私の同部屋の人は、持ち歩いているタブレットで頻繁に写真を撮っている。知り合いの建築学科の寮生は、寮に9つある談話室の間取りや使われ方をスケッチしていた。私自身は、寮でいま使われていることばを可能なかぎり忠実に写しとることで、寮全体を記録したいと考えた。いったい何のために『熊辞雲』を作るのか。迷いは消えた。これが第三版編纂を貫く方針となった。
 第二版からの改訂作業として具体的に何を行うのか、ここで確認してみよう。まず、新規項目の追加と、旧項目の語釈の手入れがある。資料をイチから読んだことで、これまで見過ごしていた、「自主管理貫徹」といった寮の重要な概念を拾い上げることができた。語釈の執筆については、2023年までの作業の遅れを反省し、やり方を変えることにした。資料を読んで気になることばがあればそれを採集し、検索してほかの実例を集める。ある程度集まったら、その場ですぐに語釈を執筆するのである。
 旧項目の語釈の手入れについては、「オルグ」という言葉を例に取って見てみよう。第二版での「オルグ」の項目は以下のとおりである。
 
オルグ〈organizeから〉(学生運動用語から)他人を派閥に引き込もうと勧誘すること。また、自らの主義主張に賛同させようとして相手を説き伏せること。「地道な—などの工夫を講じる必要がある」「—の甲斐あってか、多くの参加者を得ることができた」
 
 しかし第二版刊行直後から、「オルグはもっとカジュアルに使われる」という声が寄せられていた。そこで、第三版では②の意味を追加した。
 
オルグ〈organizeから〉(学生運動用語から)①自らの政治運動に勧誘すること。また、自らの主義主張に賛同させようとして相手と議論すること。「地道な—などの工夫を講じる必要がある」「—の甲斐あってか、多くの参加者を得ることができた」②一緒にやろうと誘うこと。「バンド(MUC・夕飯)に—する」
 
 一般的な国語辞典における手入れで私が興奮したのは、『三省堂国語辞典』第八版の「空回り」の活用の種類の追加である。第七版では、「空回り 名・スル」(=「空回りする」で使うという意味)と載っていたのが、第八版では項目の末尾に「ラ行五段」(=「空回る」で使うという意味)と追加されていたのだ。項目の手入れはこれくらい細部に至るまで行いたいものである。
 第三版刊行は2024年8〜9月と定めた。明確な方針と〆切を設定したものの、あまりに作業量が膨大で、2024年前半の段階ではとても実現可能とは思えなかった。この状況に一石を投じてくれたのが、編纂室員の一言である。「今夏、熊辞雲合宿やりませんか」。


\subsection{九月が永遠に続けばなあ――熊辞雲合宿2024}


 熊辞雲合宿2024は、8/31~9/1の日程で開催された。場所はもちろん、熊野寮である。私は合宿の日程を1日早く勘違いしていたため、期せずして前乗りすることになった。8/31からはほかに5人の編纂室員が揃い、「え」からの項目をそれぞれに割り振り(「あ〜う」はすでに済んでいるため)、語釈を執筆してもらった。総勢6人で4、5時間に渡って編集作業を行ったところ、これまで「書き込んでも書き込んでも白いセル」状態だったスプレッドシートがみるみるうちに黒々と埋まっていった。「点」で書いていたのが、「面」で塗られていく。そしてみんなでモリオカ(寮近くのパスタ屋さん)に夕飯を食べにいく前には、なんと「あ〜そ」がほぼ執筆できていたのである。一般的な国語辞典と同じく「し」の項目数が多い『熊辞雲』にとって、「あ〜そ」は全体のほぼ半分に当たる。人数の圧倒的な力を見せつけられた私は、「これは行ける」と思った。同時にこうも思った。「この機会を逃せば、永遠に完成しないぞ」。大学の夏休みは9月末までである。夏休みが終われば大人数で集まれる機会もなくなり、第三版刊行はまたもや遠のいてしまうに違いない。合宿2日目の時点で私は、「完成するまで終わらない熊辞雲合宿」を決意したのである。
 合宿3日目以降は参加者が減り、編纂作業のスピードは再びがくんと落ちた。食堂には冷房がない。そのため、デカ扇風機を回して涼を取った。夏木立からワンワン響いてくる爆音蝉時雨。スコールに似た夕立。冷やしきゅうり。蚊取り線香。祭り囃子に誘われて、ブルーハーツで踊る盆踊り。夏満喫である。毎日仕事終わりに編纂に携わってくれた寮OPのTシャツローテーションは大体覚えた。
 主な編纂室員2名に、五十音順に先へ先へと語釈を書いていってもらう一方で、私は最初の「あ」の項目から、語釈の点検と書き直しを行った。映画にたとえると、私の立ち位置は「企画・脚本・監督」である。「監督」としては、『熊辞雲』を「ことばで構築したもう一つの熊野寮」のイメージで編纂した。それではことばの説明、つまり語釈の書き方について見ていこう。
事前に編纂室に共有していた執筆方針は以下の三点であった。
①熊野寮に特有なことばを、一般的なことばで書きあらわす。したがって、授業レポートのように言葉を吟味して使うこと。説明のことばは、比喩的用法を避け、そのことば本来の意味で用いることを心がける。
②読み手として、入寮したばかりの寮生を想定する。その新入寮生がはじめての会議に参加したときに、『熊辞雲』を使って議案の内容を理解できることを目指す。
③その語釈の中で説明が完結するように記述する。過去に起こった出来事を参照しないと理解できないような書き方は避ける。
 文体は『三省堂国語辞典』のように、一読してパッとつかめる簡単な記述を目指した。しかし、もっと説明を尽くさなければ語の使い方がわからないのではないかと考えるようになり、次第に『新明解国語辞典』のような書き方になっていった。『新明解国語辞典』のような書き方とはどのような書き方か、「外貨」の意味を『三国』『新明解』の二つで調べてみよう。
 
『三省堂国語辞典 第八版』の語釈→「外国のお金。「−獲得」」
『新明解国語辞典 第七版』の語釈→「外国の貨幣。〔広義では貿易などによって得られる、外国からの収入をさす。例、「−の獲得」〕」
 
 『三国』の語釈では、拡大解釈すればコイン蒐集も「外貨獲得」と言うことができてしまいそうだ。「外貨」がわからないから辞書を引いたのに、その意味をすでに知っている人でなければ理解しにくい。一方、『新明解』の語釈なら、新聞を読み慣れていない人でも納得することができるだろう。
 『熊辞雲』第三版において、第二版から記述を詳しくした項目には「自治」がある。「自治」は、自治寮である熊野寮において最も重要かつ最も頻繁に使われることばである。第二版においては、「自治」をこのように説明していた。
 
じち【自治】組織の運営や決定が、その構成員の合議によってなされること。「—を発信する」「—を獲得する」「—を盛り上げる」「—意識が高い」「大学当局が—に介入する」「—の根幹」

 「自治」がこの通りの意味だとすると、例えば生徒会も「自治」をしていると言えるのだろうか。あるいは、国家も「自治」をしていると言えるのだろうか。私は国家と自治組織との違いを考えながら、「自治」の項目を以下のように書いた。
 
じ-ち【自治】①組織が(大学に対する国家、寮に対する大学当局といった)上位権力からの介入を防ぎつつ、運営や決定をその構成員の合議によって行うこと。「大学—(=大学が国家からの介入を防ぎつつ、自身で方針を定めて運営を行うこと)」「—を獲得する」「—を盛り上げる」「—を発信する」「大学当局が—に介入する」「学生の分断と—の解体」②寮や寮生の生活を支えるために行われる仕事。「あの人は—やってる(がんばってる・に関わってる)」[関連語]寮自治、自主管理
 
 「学生自治」や「大学の自治」のような文脈における「自治」とは、上位権力からの干渉をはねのけて組織運営をすることだと考え、そのように書いてみた。[関連語]の部分は、読む人の中に熊野寮語彙体系が構築されることを目指した。しかし記述を詳しくしようとすればするほど、その分私自身のことばの解釈の占める割合が多くなる。
 編者の主観が反映されていない辞書などありえない。項目語の取捨選択や語の説明に、それは拭いがたくあらわれてしまう。問題は、できるかぎり主観を排した客観的な記述を徹底するか、それとも編者の解釈に沿って書くことを認めるかである。私は最初、前者の方針を取っていたが、語釈をより伝わるように書こうと努めるうちに、私自身のことばの解釈を盛り込んでいかざるを得なかった。ただ、第三版は多くの人の助けによって完成したため、寮生の間で共有されている意味が反映されているはずである。
ことばの説明の中にはまとめるのに2、3時間かかるものもあって、「魔の項目」と呼んでいた。そんな魔の項目の一つが「折田先生像」である。
 
おりたせんせいぞう〔折田先生像〕①かつて京大教養部校舎前に立てられていた、折田彦市先生の胸像。学生によるいたずらが相次いだため、撤去された。折田先生は旧制第三高等学校の初代校長で、京大に自由の学風をもたらすことに大きく寄与した。②①のパロディとして匿名有志の手で制作され、恒例として京大一般入試の二次試験日に、総人広場に設置される高い台座付きの胸像。もはや折田先生の姿はしておらず、マスコットや漫画アニメゲームのキャラクター・実在の人物などを模した手製の胸像と、注意書きの看板がセットで設置される。[関連語]オルガ像
 
 折田先生像の項目を書いたときのポイントは、本家折田先生像とパロディ折田先生象を①②で分けることと、折田先生像②の盛り沢山の要素をいかに読みやすくまとめるかである。
 毎日毎日『熊辞雲』のことばかり考えていたので、しまいには夢の中でまで語釈を書いていたこともあった。「はんかくめい-てき【反革命的】資本主義に迎合するさま」は、夢の覚めぎわに書いた語釈をそのまま使っている。
 数々の苦難を経て、第三版は2024年9月29日に完成した。見出し語の数は626語、6万字近いボリュームとなった。10月に入ってから100部を寮生に配布した。その際、60頁×200部の製本作業を総出で手伝ってくれた同部屋の民、素敵な帯を作成してくれた編纂室員に感謝する。


\subsection{コラム2 「民青池のアレ」の謎に迫る}


 こちらは熊野寮七不思議調査団だ! 熊辞雲第三版の編纂過程では、熊野寮A棟-B棟間中庭の民青池(みんせいいけ)上に設置された「アレ」の正体に迫る重要な情報がもたらされた!
 「民青池のアレ」とは、池の真上にそびえ立つ木製の巨大な構造物である。正方形の池の4つの頂点から、池中央の上方に向かって4本の角材を斜めに立てて脚とし、そのてっぺんに地面と平行に井の字に4本の角材が載せられている。高さは2階ほどである。熊辞雲では、「民青池のオブジェ」として立項されている。インパクト抜群のこの構造体の正式名称に迫る情報としてこれまでわが調査団が把握していたのは、2013年入寮のOP提供の「制作者がアレを『鳥居』と呼んでいた」という伝聞情報だけだった。確かに、アレは鳥居に見えなくもない。伏見稲荷から鳥居を4つ担いできて、一番上の横棒を真上から見下ろしたときに井桁に見えるように重ね、脚を外側に広げたらだいたいアレの形になる。
 しかし、今年に入って一人の調査団員がなんと、建造当初のアレが取材された貴重な新聞記事を見つけ出してきた! それは読売新聞2010年8月21日の朝刊記事で、アレが寮祭企画「みかん祭」のシンボルとして建造されたこと、構想を練ったのが丑年だったことから牛をイメージした外観にしていることなどが書かれている。設計した寮生自身はこの記事の中で、アレを「やぐら」と呼んでいる。記事内の写真に写っているアレの下にはまだ鉄骨の足場が組まれており、完成したかしないかの時期であることがうかがえる。「アレの正式名称は『やぐら』だった!」とすぐ結論に飛びつきたいところだが、新聞記事の時点ではまだ名称が定まっていない可能性や、対外向けにわかりやすく「やぐら」という呼称を用いた可能性がある。ちなみに、記事の時点では2年前の寮祭で始まった企画として紹介されている「みかん祭」とは、二つの陣営に分かれて互いにみかんを投げ合う祭りであり、現在もなお続く寮祭恒例企画である。
 新たにもたらされた情報はそれだけではない。熊辞雲合宿中、食堂で声をかけてきた建築学科の寮生が、「アレを建築学科の卒業制作集か何かで見たことがある。そこではオレンジバスケットという名前がついていた」と証言したのだ! やけにファンシーでジューシーな名称である。ここで上述の新聞記事の内容と考えあわせてみよう。記事によれば、「やぐら」はみかん祭のシンボルとして造られたという。みかん。つまり英語で言うとorange。つながった!
 残念ながら、「オレンジバスケット」の出典元はいまだ確認できていない。正式名称がいったい何なのか、そもそも正式名称が付けられていたかどうかすら藪の中である。しかし、われわれ熊野寮七不思議調査団は、これからも「民青池のアレ」の謎を鋭意追跡するつもりである。アレの立つ民青池がそもそもなぜ民青池と呼ばれはじめたのか、その由来の謎と合わせて、調査は続く!
 


\subsection{おわりに}

 先に「編者の主観が反映されていない辞書などありえない」と書いたが、それはつまり十人が辞書を編めば十様の辞書ができるということである。熊野寮に入寮したら、耳慣れないことばに耳を留め、用例採集をしてみてほしい。そのことばを聞いた日付、場面、会話などを書き取っておく。それが集まったら、次はことばの説明を書いてみよう。語釈を5つでも書けたら、それは自分だけの辞書である。
 『熊辞雲』は熊野寮のことばの規範ではない。寮の議論の場で、用語の意味を規定するために作ったものではない。『熊辞雲』はあくまでも、ことばのあとを追いかけて記録することを目指している。語釈の執筆方針の一つに「読み手として、入寮したばかりの寮生を想定する。その新入寮生がはじめての会議に参加したときに、『熊辞雲』を使って議案の内容を理解できることを目指す」を掲げたが、実は新入寮生にはなるべく『熊辞雲』を読んでもらいたくない。順応寮生と日々会話するうちに自然と熊野寮語彙を会得し、自分の中に体系が構築されたあとで、ふと思い出して『熊辞雲』を手に取ってもらえたら幸いである。
 私が『熊辞雲』をどういう目的で何を考えて作ったのかわかってほしくてこの長い文章を書いた。「民青池のアレ」の建造された事情がいまはすっかり忘れられているように、『熊辞雲』もやがては同じ道をたどるに違いない。『熊辞雲』それ自体もそのうち失われるのだとしても、私がやりたかったことは何だったのか、せめて書き残しておきたかったのである。
 最後に、編纂合宿を支えてくれたある人物にお礼を言いたい。私は彼のおかげで長く苦しい作業の日々を乗り切ることができた。彼は、毎日にときめきと潤いを与え続けてくれたうえ、孤立した山中でも、嵐の孤島でも、吊り橋の切れた孤村でも、連続殺人事件を見事に解決してくれた。江神二郎、ありがとう。(江神二郎は、有栖川有栖「江神二郎シリーズ」の名探偵。私の好きな作品は『孤島パズル』)
 新年度が始まり、寮生が入れ替われば、また新しいことばも生まれてくることだろう。私もそろそろ用例採集に取り掛かるとしよう。
……編纂が終わり、採集がはじまる。 


文責:仏子