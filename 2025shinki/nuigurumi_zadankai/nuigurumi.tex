\documentclass[9pt,4aj]{jsarticle}
\usepackage[top=30truemm,bottom=25truemm,left=20truemm,right=20truemm]{geometry}
\usepackage{ascmac}
\usepackage{color}           
\usepackage{multicol}
\usepackage{okumacro}
\usepackage{ulem}
\usepackage{pxrubrica}
\setlength{\columnseprule}{0.4pt}
\setlength{\columnsep}{3zw}
\begin{document}
\interfootnotelinepenalty=10000
\newcommand{\talkernui}[1]{\noindent{\gtfamily \bfseries#1}:}

\section{ぬいぐるみ座談会}
12月1日、熊野寮祭3日目の昼下がり。食堂では、他の企画に参加する寮生たちが季節に似合わぬ熱気を放っていた。冬の寒さで冷めることがないのは、熊野寮に関わる誰もが1年間ずっとこの日を心待ちにしていたからだろう。しかし、今年の寮祭を楽しみにしていたのは何も人間だけではない。「ぼくたちも寮祭を楽しみたい!」と一念発起したとある寮生(ぬいぐるみ)を中心として、今、普段交わることのないぬいぐるみたちが寮内各地から集まっている。なんでもせっかく一堂に会するのだから、その証を後世に残すために座談会を開くことにしたらしい。以下は、それぞれの持ち主の寮生たち(こっちは人間)が手伝ってできあがった、座談会の記録である。

%自己紹介

\begin{itembox}[m]{\LARGE 参加者}
{\small
\talkernui{カムイ}
座談会の発起人。COLORATAのシロクマで、白くてふさふさの毛並みとつぶらな瞳が特徴的。人前に出るのが初めてなので、実はめちゃくちゃ緊張している。おっとりとして落ち着いた性格。

\talkernui{ニクスのサメ}
マリンパークニクスで2010年に生まれた。大きな口が特徴で、驚くと口を大きくあけてリアクションする。寿司が好き。一年前は大事にしていた服をなくしてしょげていた。

\talkernui{セクシー大根}
座談会へは初参加。北関東の畑から実力でここまで来た。身長80cm、直径20cmほど。たくあんみたいだねって言うと怒る。


\talkernui{サンちゃん}
座談会への参加は初めて。京都水族館3周年の時に3333円のオオサンショウウオセットとして持ち主のおうちに来た。約85センチのLLサイズ。座談会中はだいたい持ち主に抱きしめられていた。

\talkernui{くるみん}
持ち主が昨年3月末の学会のポスター発表で使うために作った「クルミホソガ」の生まれたての幼虫のぬいぐるみ。全長45センチ弱。指示棒代わりにされるほか、説明のためによく体を二つ折りにされる。お腹側にベルトを通すための仕掛けがあり、ショルダーバックのベルトに通して連れ歩かれることも。学会では人気者だった(たぶん)。

\talkernui{モドキ}
手乗りサイズのぬいぐるみキーホルダー。てんとう虫みたいな見た目だが人の顔をしている。

\talkernui{リュカ}
ピンク色のテディベア。持ち主に(背負ったら絶対可愛いから)天使の羽根を与えられ、(抱きしめるときに邪魔になってきたので)もぎ取られた過去がある。身長60cm、座高40cm。

\talkernui{しまじろう}
本名は島田心魔死路卯

\talkernui{ちびお}
持ち主が小学生の時に昆虫のテーマパークで購入した、持ち主一家では第4世代の中型のカブトムシのぬいぐるみ(7匹目!)。モリオンのような\ruby{円}{つぶら}な黒い瞳と滑らかな手触りの明るい色をした毛並みが特徴で、スカートのような服を着せられていた時期もあった。持ち主の所望とあらばどこにでも連れて行かれる。

\talkernui{アルパカ}
持ち主が中学生の時に林間学校に行った先で貰った地元の方の手作りぬいぐるみ。白い毛並みの上に乗る\ruby{円}{つぶら}な黒い瞳と頭頂部の褐色をした本物のアルパカの毛がアクセント。持ち主一家のぬいぐるみ達の中では比較的新参者である為キャラ付けと使用経験は共に浅いが、ちびおと並べられることは多い。

\talkernui{IKEAサメ}
ご存知IKEAのサメ。どーもくんのような大きい口を持つ丸っこいヨシキリザメのぬいぐるみ(体長約90cm)。まだ幼いのでわがまま放題で、なんでもかじりたいお年頃。最近は釣り動画の影響でかじる前に叩いて〆ようとするようになった。語尾も一人称も「サメ」。

\talkernui{ひま}
なんでも肯定してくれるコウテイペンギンことコウペンちゃんの「お日さまだ〜いすき!ひまわりコウペンちゃんマスコット」シリーズ。手に持っている向日葵は武器にもなるしフライ返しにもなる。君のことが大好き。

\talkernui{ベレー}
コウペンちゃんの「ベレー帽とマフラーで一緒におでかけマスコット」シリーズ。趣味はムエタイ。最近の悩みはベレー帽を還暦祝いの帽子と間違われること。寒い日でも君とおでかけできるなら嬉しいらしい。君のことが大好き。

\talkernui{イッチ}
コウペンちゃんの「いちごさんのケープなマスコット」シリーズ。趣味は5chでレスバすること。レスバにはよく負けて悔し涙を流す。最近、ケープが開いてしまい、便座カバーを被っているみたいになってしまった。君のことはちょっと好き。

\talkernui{土偶}
重要文化財に指定された亀ヶ岡遺跡出土の遮光器土偶の頭部。現在、いち寮生の卒論執筆を部屋で監視している。ほどよい柔らかさ。

\talkernui{レジロック}
むかし ひとに ふういんされた いわの ポケモン。ポケモンセンター おおさかで であった。ぬいぐるみに なったからって つごうよく にほんごを しゃべっては くれない。
}
\end{itembox}

%自己紹介
\begin{multicols}{2}
\noindent{\uuline{\Large\textbf{自己紹介\\}}}
\talkernui{カムイ}
まずは自己紹介から始めます。皆さんのお名前と、どこに住んでるか、などを教えてください。まず僕から。普段は持ち主の居室に住んでます、カムイと言います。

\talkernui{ニクスのサメ}
さめさんです。普段は自分の部屋に住んでいて、14歳です。食北のサメ\footnote{食北とは、食堂の北側にある談話スペースのこと。そこには\textgt{IKEAサメ}とは別の、IKEA出身のサメのぬいぐるみが住んでいる。}にスケールが負けちゃうけど頑張って生きてます。

\talkernui{セクシー大根}
名前はないですが、セクシー大根でやらせてもらってるよ~ん。チュッ、チュッ。年齢は、1歳と4か月ぐらい。自分の部屋のベッドでいつも寝てます。よろしくね~。

\talkernui{サンちゃんの持ち主}
こちらが私の居室に住んでいるオオサンショウウオのサンちゃんと言います。京都水族館3周年の時に、オオサンショウウオセットが3333円で売っており、オオサンショウウオのバッチとか、八つ橋とか、そういうものと一緒にわたしのおうちに来て、寮まで来ています。主に冬場のおなかや背中を温めるお仕事をしてくださっており、抱き心地が非常にいいです。

\talkernui{サンちゃんとくるみんの持ち主}
こちらは、お名前をくるみんと言いまして、去年の3月に生まれました。私が作りました。私の研究している虫を学会のポスター発表で説明するために、作ったという経緯で生まれた子です。普段この子は研究室で愛でられております。ポスター発表の度に指示棒としても活躍してくれてます。

\talkernui{モドキ}
ぼくはてんとう虫みたいな見た目をしてるんですけど、手足が4本しかなくて、本当はてんとう虫ではないので飼い主にモドキといわれています。普段は本棚の隅に座っています。こちらは最近モドキの飼い主の友達が作ってくれたちびもどきです。すでに切っても切れない関係になっています。よろしくお願いします。

\talkernui{リュカ}
リュカです。持ち主との付き合いは5年ぐらいになるかもしれない。あれはたしか...、四条大宮のあたりでリサイクルショップというところにずっといたんですね。今の持ち主の人がアルバイトを転々としてた時期で、その時は子供科学実験教室の面接を受けに行く日だったんです。時間を余らせて、なんかリサイクルショップで掘り出し物を探してしまったらしいんです。そしたらぼくを見つけてくれて。リュカは男の子の名前です。一時期持ち主に溺愛されすぎて、メルカリで発掘された「新生児 天使の羽 コスプレ」で検索できる、なんか最近そういう人多いじゃないですか。子供の写真撮るときに。ゴムになってて、両肩にかけて、皆さんは新生児の写真をとるらしいんですけど。絶妙に新生児っぽいサイズで新生児らしからぬところどころのサイズの関係がありまして、肩のあたりに通す紐を全部首に通されて、その代償として天使の羽を付けられて、ずっと過ごしてました。最近は抱き心地の問題で外されて、ちょっと楽ちんになりました。よろしくお願いします。

\talkernui{しまじろうの持ち主}
しまちゃんあんまちょっと今機嫌よくないから...。

\talkernui{しまじろう}
{\large 名前はしまじろう!!!年齢はっ!!56歳!!居室に住んでる!!以上だ!!!}(こたつを叩きながら)\footnote{文字だけでは面白さが伝わらなさそうなので補足。大の大人(大学生)が頑張っておじさんの野太い声をまねながら自己紹介している様子を想像してみてほしい。以下、\textgt{しまじろう}はそのような声で話す。}

\talkernui{ちびお}
僕はちびおという名前です。10年ぐらい前に福島県にあるムシムシランドという名前の昆虫のテーマパークで現在の持ち主によって購入されました。普段は夏休みの持ち主の帰省に伴って東京の実家からここ熊野寮へ持ち出されて、長いことローテーブルの上にほってかれてたんですけど、持ち主の扱いの荒さからいつのころからか床の上に落っこちて、長いこと参考書や紙などの下敷きになっておりました。何卒よろしくお願いいたします。

\talkernui{アルパカ}
私は普通にアルパカと呼ばれております。持ち主が中学2年生の時に林間学校に行った際に、その場所がアルパカの飼育で有名な場所でして、アルパカに関連した製品を作成していたんですね。で、持ち主が泊まった先で記念だからと言ってもらいました。長年扱われている割にはなかなか黄ばんでこないのがすごいなぁと思っております。ちなみにこの頭の毛は本物のアルパカの毛です。結構触り心地いいです。よろしくお願いします。

\talkernui{カムイ}
皆さんありがとうございます。一応あともう一人後からくると聞いている子がいるので、その子が来たらまた自己紹介してもらいましょう。

%みんなどんな生活しているか
\noindent{\uuline{\Large\textbf{みんなの生活スタイル\\}}}
\talkernui{カムイ}
未来のぬいぐるみ寮生たちに向けて、熊野寮がどんな感じか、皆さんがどんな生活しているか、とかを聞いていこうと思います。皆さんどんな生活スタイルですか?僕は基本普段ベッドで寝てるんですけど、

\talkernui{リュカ}
ぼくも同じです。

\talkernui{カムイ}
持ち主の人とこれやってるよ、とかこういう趣味あるよ、とか。

\talkernui{ちびおの持ち主}
アルパカとちびおは熊野寮に来てからずっとローテーブルの上に置かれ続けるだけの存在だったんですけど、実家ではアルパカは机の引き出しに延々としまわれ続け日の目を見ることもなく、長いこと放置されておりました。ちびおに関しましては、同じように引き出しの中に長いこと収められていたんですけど、小学校高学年から中学生ぐらいまでは持ち主にちゃんと溺愛されまして,寝室のある生活をしておりました。朝になったらその場所から出され、目立つところに置かれ、夜になったらベッドで寝かせられるという。ちなみに服とかも着せたことがあって、古着を着せて縫うなり輪ゴムでいろんな個所を結んだりしました。

\talkernui{ニクスのサメ}
なんかzoom参加が。

(\textgt{レジロック}が参加)

\talkernui{レジロックの持ち主}
レジロック君がzoom参加してくれてます。「ビビビビー」という感じでしかしゃべれないのでちょっと通訳します。それじゃあ、レジロック、自己紹介してあげて。

\talkernui{レジロック}
ゲゲ、ゲゲ、ゲゲビビ、ゲ、ゲ、ゲゲ、ゲゲ、ゲゲ、ゲゲ、ビ、ビ、ゲゲ、ゲゲ、ゲゲ。

\talkernui{レジロックの持ち主}
あのー、ちょっと緊張してるので、レジロックですとしか結局いえてないです。めちゃくちゃ長くしゃべってる感じなのに。

\talkernui{レジロック}
ベべ、ベ、ベ、ベベ、ベ、ベ、ベベ、ベ、ベベベベ。

\talkernui{レジロックの持ち主}
なんか飼い主がエジプトに旅行してるときに、遺跡の中で、出会った。

\talkernui{一同}
すご~!

\talkernui{サンちゃん}
エジプト語しゃべれるんですか?

\talkernui{レジロックの持ち主}
エジプト語しゃべれる?

\talkernui{リュカ}
アラビア語じゃない?

\talkernui{サンちゃん}
あ、すいません。

\talkernui{リュカ}
いえいえ。小池百合子が\footnote{小池百合子がカイロ大学出身でアラビア語学んでいたはずなのに意外と話せてなくね?と一時期話題になった。ぬいぐるみも社会で生きているので、こういった卑近な話も意外と知っている。}。

\talkernui{レジロックの持ち主}
www。「サラーム」だけ言えるそうです。挨拶だけできるそうです。

\talkernui{リュカ}
ビビービですか?

\talkernui{レジロックの持ち主}
ビビービ、ビビービ、ビビービ。あ、ちょっと消えちゃう...。あ、ちょっと電波が悪いようです。

(\textgt{IKEAサメ}が参加)

\talkernui{IKEAサメ}
けっこうおおきい企画サメ。

\talkernui{カムイ}
\textgt{IKEAサメ}さん自己紹介お願いします。

\talkernui{IKEAサメ}
サメサメ。好物はお魚サメ。おさかなが好きサメ。いつも部屋のベッドでねてるサメ。よろしくサメ。

\talkernui{一同}
(拍手)

\talkernui{カムイ}
僕の方から皆さんに聞きたいことがあるんですけど...。単刀直入に言うと熊野寮って汚くないですか?

\talkernui{一同}
www

\talkernui{しまじろう}
汚ねぇ!!

\talkernui{カムイ}
そう。汚いんですけど、ぬいぐるみの皆さんはいつもどういう風に思ってます\footnote{\textgt{カムイ}は白くて毛並みがふわふわなので、寮の床に落ちるのが物凄く怖い。}?

\talkernui{ニクスのサメ}
帰省の度にお風呂に入ってます。

\talkernui{一同}
へぇ~。すごい。

\talkernui{ニクスのサメ}
ちゃんと洗ってもらってます。きちんと乾かして、こっちに戻ってくるようにしてます。こっちに来る前からずっと、3か月にいっぺんぐらいの頻度でお風呂に入って、お姉ちゃん以外のお友達とも一緒にお風呂に入ってきれいにしてもらってます。こっちに来てからまぁ汚くはあるけど、そこそこね、ゆっくりと、生きています。

\talkernui{サンちゃん}
やっぱり熊野寮で生きる上で床に落ちないのが最も大切なことでして、

\talkernui{カムイ}
そう!

\talkernui{サンちゃん}
床に落ちると、床は汚いので汚れてしまうのですが、すこし高めの位置にいれば大体ホコリは被らないというのがありますね。最初はベッドの中に入ることによって床に落ちないという対策をとっていたのですが、ベッドの柵に挟まって体が変形してしまうという事件が起きて、それ以来ベッドの横から出され、机わきによく置かれています。落ちないのが一番の対策だと思います。

\talkernui{カムイ}
サンちゃんはよくお風呂とか入ったりするんですか\footnote{あまり関係ないが、寮生(人間)で数日に一度しかお風呂に入らない人はたまにいるらしい。持ち主が言うには、熊野寮で「お風呂に入ってますか?」と聞くことは、「最近ちゃんと授業行ってる?」ぐらいのノリであるそうだ。}?

\talkernui{サンちゃん}
やだ~。生まれてこの方入ったことがないですぅ~。

\talkernui{カムイ}
それは汚れたことがないからですか?

\talkernui{サンちゃん}
今のところ汚れているという認識を飼い主にされたことがなくって、正直洗った後乾かずに生乾き臭がするのが一番怖いなーっていう気持ちもあって、

\talkernui{リュカ}
分かります。あのー、大きいと特に乾燥の問題が発生するというか。ぼくは出自がリサイクルショップなので、お迎えされた直後にお風呂にザボンと入れられて、わしゃわしゃわしゃってやって、色が何トーンか落ちました。だからお風呂に入る大切さはよくわかったんですけど、たまにまた思い立って入れられると、ピンクがちょっと薄くなっちゃうんで、うーん。悩ましいところなんですけど、汚れないことを大切にしています。

\talkernui{IKEAサメ}
サメはたまに屋上につるされてるサメ。よいサメ。

\talkernui{レジロック}
ビビビビビ、ビビビビ、ビビビビ、ビビビビ。

\talkernui{レジロックの持ち主}
なんか、僕はキャラ的に汚れてる方が嬉しい。

\talkernui{一同}
あ~。

\talkernui{サンちゃん}
いやでもほこりは嫌じゃないですか?

\talkernui{レジロックの持ち主}
ほこりは、

\talkernui{レジロック}
ビビ、ビビ、

\talkernui{レジロックの持ち主}
めちゃくちゃ嫌だそうです。

\talkernui{レジロック}
ビビビビ、ビビ、ビビビビビビビビ。

\talkernui{レジロックの持ち主}
僕は岩タイプなのでお風呂には絶対に入りたくない。

\talkernui{カムイ}
負けちゃいますもんね~。

\talkernui{ちびお}
洗濯について僕一つ思い出すことがあります。僕を買うにあたって、なんか商品タグを読んで、その中に洗濯すると色落ちする恐れがあります、という説明があったんですね。で小学生の頃の持ち主は、色落ちすることを非常に恐れて、ぬいぐるみが水に触れないように細心の注意を払っておりまして、ほとんど洗濯したことはなかったです。僕以外の他のぬいぐるみもそうです。まぁ、だがしかし。僕の飼い主は...

\talkernui{カムイ}
...しまじろうさんお酒とかいります?(しまじろうがイライラしていたので)

\talkernui{しまじろう}
{\large ヤニ!ヤニ!ヤニが足りないっ\footnote{たばこが足りていないという意味。}!!}

\talkernui{一同}
www

\talkernui{セクシー大根}
鬼殺し\footnote{日本酒の一種。コンビニとかによく売っているそう。}とかのんでるんかね。

\talkernui{カムイ}
確かにwww

\talkernui{しまじろうの持ち主}
酒よりヤニでしょ。

\talkernui{カムイ}
確かにタバコ吸ってらっしゃいましたね\footnote{寮内にある雑記帳に、朝起きたら\textgt{しまじろう}がたばこ臭かった、という趣旨の記述があった。\textgt{カムイ}は持ち主から聞いていて、リサーチ済みであった。}。

\talkernui{しまじろう}
{\large イライラするなぁ!}

\talkernui{サンちゃん}
どの銘柄が好きですか?

\talkernui{しまじろう}
{\large わかば\footnote{たばこの銘柄の一種。結構昔からあるそう。}!!!}

\talkernui{一同}
www

\talkernui{カムイ}
ありがとうございまーす。(拍手)

(\textgt{カムイ}が\textgt{しまじろう}にガン飛ばされる)

\talkernui{カムイ}
ちょっと建物内禁煙なので\footnote{熊野寮には喫煙所があり、そこ以外は禁煙である。}、ここでのおタバコはご遠慮ください(汗)。

%どこかお出かけした?
\noindent{\uuline{\Large\textbf{持ち主のひととお出かけしたところ}}}\\
\talkernui{カムイ}
続いて、僕あんまり寮の外に出たことないんですけど、持ち主の方とどこかお出かけしたことある方とかいらっしゃいますか?

\talkernui{IKEAのサメ}
サメは大阪のIKEAしゅっしんサメ。大阪のIKEAに行ったときについでに水族館に行ったから、IKEAバッグからかお出しておいしそうなおさかなたくさん見たサメ。あと、丸げんラーメンにも行ったことがあるサメ。ブロックかいぎ\footnote{月2回開かれる会議。熊野寮はフロアごとにブロックという単位でくくられており、仕事などをブロックの中で回す。ブロック会議では、ブロックの構成員たちが集まってイベントの周知をしたり、寮の方針などについて話し合ったりする。毎回日付が変わるぐらいまでかかるので、終わった後のテンションのままラーメンを食べに行く界隈もあると持ち主が言っていた。}後に行ったことがあるサメ。

\talkernui{セクシー大根}
自治意識もある。

\talkernui{IKEAのサメ}
そんなにサメ。ご飯くれるならなんでもいいサメ。じゃあ小ざかな。小ざかなの番サメ。

\talkernui{ニクスのサメ}
んあ?んあ?(\textgt{IKEAのサメ}に威張る)

\talkernui{カムイ}
こらこら。どうどうどう\footnote{相手はサメだが。}。(\textgt{カムイ}が二人を諫める)

\talkernui{ニクスのサメ}
ぼくは、飼い主が小学校2年生の5月ぐらいに出会っているので、いろんなところに行ってきました。そもそもぼくは北海道の苫小牧で出会いました。だから、基本的な北海道は全部行きました。函館から、札幌小樽は行ったし、旭川も行って十勝も行ったし、知床もいきました。あとは、飼い主の空手の遠征に従って岐阜や東京などにも行きました。あとは、この前青森まで18きっぷの旅にも同行しました。日本中いろんなところに行っていろんなものを見てきたサメです。雑魚。(\textgt{IKEAのサメ}を指さしながら)

(\textgt{IKEAのサメ}と\textgt{ニクスのサメ}が揉み合う)

\talkernui{カムイ}
どうどうどうどう。暴れない、暴れない。

(\textgt{IKEAのサメ}が\textgt{カムイ}を指さす)

\talkernui{カムイ}
いやいやいやいや。

\talkernui{リュカ}
責任転嫁www

\talkernui{レジロック}
ビビビビー、ビビ。

\talkernui{レジロックの持ち主}
ポケモンバトルのルールにのっとって解決しましょう。

\talkernui{IKEAのサメ}
(\textgt{レジロック}を指して)えらそうサメ。バトルしたらサメがかっちゃうサメ。いわタイプはざこサメ。

\talkernui{一同}
www

\talkernui{カムイ}
くるみんさんとか学会発表とかよくいってらっしゃるアカデミックなぬいぐるみで。

\talkernui{くるみん}
そうなんです。私実は学会で、ポスター発表する時の説明用として作られまして。去年の3月に作られたんですけど、ポスター発表1回、施設での発表みたいなものにも1回、さらに国際学会に来た方々へのご案内の時とかにも1回、他にもいろんな場所で使われてますね。まず学会で仙台に行きまして、仙台の伊達政宗公の銅像の横で謎にこういう形\footnote{伊達政宗の兜の形を実演している。}でぬい撮りをされ、なんかその後の懇親会でも「目立つからいいんじゃない?」の言葉とともにショルダーバッグのベルトに通されてずっと飼い主についていって、懇親会ではバッタの仮面を被った人に声を掛けられたりしました。すごく近しいものを感じたという。そういうことで、私いろんな場所に遠征しておりまして、いろんな方々と会っております。でもまだ1歳にもなっていないので、これからもずっと研究施設にてかわいがられていく所存でございます。

\talkernui{モドキ}
ぼくはよく飼い主がぬい撮りに連れて行ってくれることがあります。ちびモドキを作ってくれた飼い主のお友達がすごくぬい撮りが上手で、一緒に遊びに行くときによくぼくも連れて行ってもらいます。京都だと金閣寺とか観光に行きました。でも飼い主はぬい撮りがへたくそなので、風景かぼくかのどっちかにしかピントを合わせられなくて、ちょっと不満を感じています。この前からふね屋\footnote{京都の喫茶店。ジャンボなんちゃらパフェというめちゃくちゃ大きなパフェで有名。}の唐揚げパフェみたいなおもしろパフェを一緒に食べに行きました。この後の巨大パフェを食べる寮祭企画も一緒に行きたいです。でも飼い主がぼくか巨大パフェのどっちかにしかピントを合わせられないんじゃないかって心配してます。なんでもっと飼い主にぬい撮り上手くなってほしいです。

\talkernui{カムイ}
リュカさんとしまちゃん?しまじろうさん?は、

(\textgt{カムイ}が\textgt{しまじろう}にガン飛ばされる)

\talkernui{カムイ}
怖い怖い怖い怖いwww

\talkernui{リュカ}
今お時間よろしいですか?

\talkernui{しまじろう}
いーよー。

\talkernui{リュカ}
しまじろうさんは普段どういう風に暮らされているんでしょうか。

\talkernui{しまじろう}
{\large もう外泊しまくりっ!}

\talkernui{一同}
www

\talkernui{しまじろう}
ぜんっぜん部屋居ない!

\talkernui{リュカ}
どこに泊まられてるんですか?ご結婚とかされてるんですか?

\talkernui{しまじろう}
{\large してた!}

\talkernui{一同}
過去形!?

\talkernui{カムイ}
そういえばそうだった。雑記帳に書いてらっしゃいましたよね\footnote{こちらも\textgt{カムイ}の持ち主と一緒にリサーチ済み。}?

\talkernui{しまじろうの持ち主}
そういえば書いたかもしれない。あ、思い出して泣いてます。

\talkernui{一同}
あ~。

\talkernui{リュカ}
こんなことを聞いてごめんなさい。

\talkernui{しまじろう}
オッケー。

\talkernui{リュカ}
何年ぐらい前の話なんですか?

\talkernui{しまじろう}
うーん、20年前ぐらい。

\talkernui{リュカ}
お子さんとか、

(\textgt{しまじろう}が鼻をすする)

\talkernui{しまじろうの持ち主}
あ、泣いちゃった~。

\talkernui{一同}
www

\talkernui{しまじろうの持ち主}
思い出しちゃいました。

\talkernui{リュカ}
でも身なりとかを見ると、だいぶ苦労されてるようだから、慰謝料とか養育費とかは大丈夫ですか?残らせてませんか?

\talkernui{しまじろう}
まあ何とか。何とかしました。

\talkernui{カムイ}
何の仕事してらっしゃるんですか?

\talkernui{しまじろうの持ち主}
しまちゃんなんの仕事だっけ?

(\textgt{しまじろう}が首をかしげる)

\talkernui{カムイ}
あんまり言えないやつ?

\talkernui{しまじろうの持ち主}
あんまり、あんまり。時々漁船乗せられちゃったりするやつだよね。

\talkernui{リュカ}
蟹工船?

\talkernui{しまじろうの持ち主}
そう。蟹工船乗せられちゃったよね。時々。ね。大変だったね。結構ね。

\talkernui{カムイ}
頑張ってください。(小並感)

\talkernui{レジロックの持ち主}
途中参加したい方が。

\talkernui{???}
いいんですか!?

(\textgt{コウペンちゃん}3人が参加)

\talkernui{コウペンちゃんたちの持ち主}
京都女子大学コウペンちゃん同好会の、なんでも肯定してくれるコウペンちゃんです。この子がひまわりのひまちゃんです。この子がベレー帽のベレーちゃんです。ポーズがムエタイっぽいので家族からムエタイって呼ばれてて、彼からは還暦くんって呼ばれてます。

\talkernui{一同}
還暦くん!?

\talkernui{コウペンちゃんの持ち主}
かわいそうに。この子はイチゴなのでイッチさんです。趣味は5ちゃんねるでレスバすることです。この子もともとはここがこうだったんですけど\footnote{ケープが閉じていた。}、顔が隠れちゃうから糸をとったら開くようになっちゃって、飛べるようになったっていう子です。ちょっと一発芸見てもらおうか?

\talkernui{イッチ}
見てください。

(飛び上がり、その場で頭をこちらに向けてケープをめいっぱい広げて)

\talkernui{イッチ}
便座カバー!

\talkernui{一同}
wwwww(\textgt{しまじろう}大笑い)

\talkernui{イッチ}
しょうもない一芸を。大先輩に気を遣わせちゃいました。

%京都のおすすめスポット
\noindent{\uuline{\Large\textbf{京都のおすすめスポット\\}}}
\talkernui{カムイ}
では、京都在住の方でも寮生の方でもいいんですけど、京都周辺でおすすめのスポットとかありますか?持ち主の人とここ行ってみるといいよとか、ここよかったよ、といったスポットがあれば。

\talkernui{レジロック}
ビービービー。

\talkernui{レジロックの持ち主}
あのー、龍安寺というところです。

\talkernui{一同}
龍安寺www

\talkernui{レジロックの持ち主}
石にシンパシー感じてるんでしょうね。枯山水の前でのんびりしていました。

\talkernui{一同}
あ~。いい。

\talkernui{サンちゃん}
京都水族館ですよ。やっぱ。京都水族館に行ってください。私よりも大きい、なんなら人間よりもデカい子から、手のひらに乗る程度のサイズまで、多種多様なサンショウウオのぬいぐるみが、もうサンショウウオだけでワンフロア掌握してるレベルでたくさんいます。

\talkernui{IKEAのサメ}
それっておいしいサメ?

\talkernui{サンちゃん}
我々は弱毒を持っており、体表面を触った状態で粘膜に触ると多少何かが起こるかもしれない\footnote{\textgt{サンちゃんの持ち主}曰く、「サンちゃんはオオサンショウウオの体表面の粘液に毒が含まれていると認識していますが、後から調べたところ、オオサンショウウオに毒はない、という記述もあったので、サンちゃんの「毒」ジャッジが甘い可能性があります。詳しくは自分で調べてください」だそうだ。}。

(恐怖に震える\textgt{IKEAのサメ})

\talkernui{カムイ}
\textgt{モドキ}さん金閣寺とか行かれてましたよね。他にも観光とか行かれたりしてるんですか?

\talkernui{モドキ}
京都で言ったところはどこだろうな。でも、飼い主が言うには京都って鴨川沿いを自転車で走ったりするだけで気持ちいいし、春は桜きれいだし、紅葉きれいなところもあるし、だからぼくをカバンにつけてくれたことがあるんですよね。でも2日か3日ぐらいで顔が黒くなってしまって、過保護な飼い主はもうぼくをカバンに付けてくれなくなってしまったんですよ。

\talkernui{サンちゃん}
お顔は戻ったんですか...?

\talkernui{モドキ}
その時お風呂に入れてもらいました。でももうあんまり連れて行ってもらえなくなっちゃって。悲しい。

\talkernui{サンちゃん}
最近透明のカバーとかあったりしますね。

\talkernui{モドキ}
実は飼い主それもやってくれたんですけど、顔がつぶれちゃって、かわいそうだからという理由でずっと本棚の端っこに座ってることになりました。

\talkernui{カムイ}
ちなみにしまじろう...パイセン\footnote{去年(2024年度)の入寮パンフレットを参照。}?おすすめの場所ありますか?

\talkernui{しまじろうの持ち主}
しまちゃんおすすめの場所あるでしょ?

\talkernui{しまじろう}
{\large (ドンッ)木屋町!木屋町最高!}

\talkernui{一同}
www

\talkernui{しまじろう}
あと普通に、法然院とか好き。

\talkernui{リュカ}
落ち着いた一面www

\talkernui{カムイ}
おすすめの居酒屋とか。

\talkernui{しまじろう}
おすすめの居酒屋は...、もう意識が飛んでるから分からない。

\talkernui{一同}
www

\talkernui{ちびお}
君の持ってる携帯に検索履歴か何かあるんじゃないですか?

\talkernui{しまじろう}
ガラケーなんだよぉ。

\talkernui{しまじろう}
ちなみにこれ(右胸についている丸いフェルト)は無線。

\talkernui{カムイ}
それは持ち主の方と連絡する用?

\talkernui{しまじろうの持ち主}
あ、いろんなところと連絡する・・・。

\talkernui{カムイ}
あー...。

\talkernui{しまじろうの持ち主}
まぁまぁまぁまぁ。

\talkernui{リュカ}
多分あんまり聞かない方がいい。

\noindent{\uuline{\Large\textbf{持ち主が熊野寮に興味を持っている\\}}}\\
\noindent{\uuline{\Large\textbf{おともだちへ\\}}}
\talkernui{カムイ}
名残惜しいですが、この座談会もそろそろ終わりに入ります。寮生の方\footnote{そろそろ忘れかけているころかもしれないが、「寮生の方」はぬいぐるみのことを指す。}は、この座談会を読んでいる方に向けて、熊野寮に住むとこういういいことがあるよ、ということとか、寮外生の方とかもしいたら、熊野寮来てここがよかったよとか、他のぬいぐるみの方向けに教えてもらえるとありがたいです。

\talkernui{カムイ}
それではまず僕から。持ち主が高校生だったころまでは実家以外の他のぬいぐるみとは交流がなかったので、やっぱりこういう他のぬいぐるみ寮生たちと交流できるところは熊野寮のいいところだなと思います。

\talkernui{IKEAのサメ}
サメはいろんな談話室に行っていろんな人にかわいがられててとてもたのしいサメ。ゲームで遊んだり、マンガよんだり、スマホいじったり、いろいろして、楽しく毎日すごしてるサメ。

\talkernui{サンちゃん}
おすすめのマンガは何ですか?

\talkernui{IKEAのサメ}
ハンターハンターおもしろいサメ。でも文字数おおいからさいきん読んでないサメ。絵だけ読んでるサメ。あとは...、他のぬいぐるみいっぱいいるから、かじりがいがあってたのしいサメ。あれも、これも。

(\textgt{ニクスのサメ}と\textgt{カムイ}を見る)

\talkernui{カムイ}
おいしくないよ?あっ、

(\textgt{カムイ}が\textgt{IKEAのサメ}に噛みつかれる\footnote{ちょっと痛かった...。})

\talkernui{IKEAのサメ}
おいしいサメ。みんなも来たら、おいしくてたのしいと思うサメ。

\talkernui{ニクスのサメ}
いろんなぬいぐるみと交流できるし、ぬいぐるみの愛され方がみんな各々で違っているのがなかなか面白いですね。

\talkernui{セクシー大根}
全然話せなかったんですけど、私は北関東の片田舎から来たので、あんまり外の世界は知らなくて、日本中からいろんな人が来ていろんな話を聞けるのが嬉しいかなあ。あとはやっぱりコンパとか自治寮でしかないことっていっぱいあると思っているので、そうした活動を通していろんな人と友達になれたら楽しいと思います。

\talkernui{カムイ}
コウテイペンギンさんたちは多分熊野寮に来たの初めてですよね。

\talkernui{イッチ}
めったなことは言えないんだなー、あまりなー。でも色々とたまに同好会に政治思想がすごいのも確かにいるんだなー。クメール・ルージュ同好会とかが存在するんだなー。

\talkernui{カムイ}
ありがとうございます。レジロックさんはzoom上ですけどなにかありますか?

\talkernui{レジロックの持ち主}
多分寮外生なんですけど。レジロックなんかある?

\talkernui{レジロック}
ビビビビ、ビー、ビー、ビビ―、ビー、ビー、ビー、ビー、ビビ―、ビー、ビー、ビビ―、バー、ビビ―、ビー、ビー、ビー、ビビビビ、ビー、ビー、ビービービービ、ビビ、ビー。

\talkernui{レジロックの持ち主}
我々は長い時間洞窟に押し込められてきた。外に出て人と関わる楽しみを知った。よって人と関わる機会のあるこういった空間は私たちにとって貴重なものだ。だそうです。

\talkernui{一同}
おおー。

\talkernui{サンちゃん}
やっぱりぬいぐるみ寮生に選ばれるにはまず、寮に入った飼い主に飼われるか、寮生のおうちから持ってこられるかの2択だと思うんですね。で、寮生のおうちからの話なんですが、私は寮生のおうちから持ってこられたタイプなんですが、寮って非常に汚いので、大体ぬいぐるみは持ってこないことがきっと大多数だと思うんですよ。実際私のおうちには、私よりも先輩のぬいぐるみがたんまりいるんですけど、実は私以外大体持ってこられてなくて。なんでわたしだけ持ってこられたかというと、抱っこした時の抱き心地がよく、寒いときに椅子の後ろに置いて腰を温めるみたいな役目があるからということで連れてこられたんですが、普段実家にいるときだとたくさんのぬいぐるみのうちの一人だったのがここだと一人だけみたいなのがちょっと嬉しいポイントだったりするので。そういう意味でもぬいぐるみ寮生さんたちはこう、汚い場所ではありますがここに選ばれたことに誇りをもって生きてゆきましょうということで。以上です。

\talkernui{カムイ}
ありがとうございます。くるみんさんも。

\talkernui{くるみん}
私なんですけど、寮で生まれたんですが普段は研究室にずっと吊り下げられているので...。寮で作られるためのアドバイスをしますと、まず100均に行くとフェルトが売っております。裁縫セットみたいなのを実家から持ってくると、縫うことができます。それでですね、寮内に転がっている綿製品みたいなものがあるんですね。例えばわたくしの持ち主の部屋には先住民が残した謎のクッションみたいなのがありまして、持ち主使ってなかったんですよ。それは今わたくしの糧となっております。そういう風にすると綿を購入する必要がないので、実質100円から作ることができるんです。これは寮のいいところですね。あといらない綿製品ありませんかと全寮ラインとかで声をかけるとなんと綿製品が無限に入ってくるということすらある。ということで、綿の足りない皆さん、中に綿を入れましょう\footnote{献血?どちらかというと綿は内蔵に近いかも。}。以上です。

\talkernui{モドキ}
熊野寮のいいところは...、飼い主さんが寮食美味しい寮食最高と言えと言ってきます。でもぼくとしては、安くておいしい寮食があることで飼い主があまり外食しないからぬい撮りの機会が減ってるんじゃないかなって思います。寮食で写真を撮ってほしいです。あとは寮生のみんなが優しいからぼくとちびモドキのことをちやほやしてくれるのがすごくいいです。みんなもきてね~。

\talkernui{リュカ}
持ち主とずっとそばにいたいだろ?選ばれるぬいぐるみになれっ(キリッ)。それだけ。

\talkernui{一同}
www

\talkernui{サンちゃん}
今から新しい子が。

(\textgt{土偶}登場)

\talkernui{サンちゃん}
自己紹介からメッセージまで一気にやってもらって。

\talkernui{土偶}
ちょっとまだ考えられてないから後でで。

\talkernui{サンちゃん}
じろうさんに先にやってもらおう。

\talkernui{しまじろう}
今まで奥さんに逃げられちゃったりとか、仕事で裏切られたりして、人のこと信じられなかったけど、寮に入って同部屋にぬいぐるみがいて、そのぬいぐるみが飼い主に似てすごく信頼できる人で、初めてそういう人に出会えたのが熊野寮だったから、自分は寮に来てほんとによかったとは思っています。

\talkernui{カムイ}
ありがとうございます。深みがある。

\talkernui{アルパカ}
熊野寮は非常に人数規模が大きい場所であるために、大概なんらか一つの趣味を掲げれば一つのサークルができるぐらいいい人数の人が集まってきてくれます。そしてその人たちとそのテーマについて交流する機会が得られるのも熊野寮ならではだと思います。実際わたくしの持ち主もここ熊野寮に来るまでは家族でない人とぬいぐるみについて交流することは一度もありませんでした。ここにきてより想像以上の多くの方々が様々な種類であったり時期に購入されたぬいぐるみを持っていて、ぬいぐるみの多様性であったり、そういったものを感じました。私の持ち主は自分の好きな生物に合わせて購入するぬいぐるみを選んでいたので、自分の偏狭さというか、と言ってもぬいぐるみを買ったのはほとんど小学校時代なんですけども、当時の自分の視野の狭さ、というと聞こえが悪いですけど。

\talkernui{ちびお}
様々な交流を通じて、ぬいぐるみ間での交流の場をどんどん広げていってほしいなあと思います。ぬいぐるみって基本的に持ち主以外の方と顔を合わせるというか、触れ合う機会がそうないし、こういった珍しい機会は寮を出たら一生得られないでしょうから寮の素晴らしい環境を最大限利用してほしいなという気持ちです。そして、後代のペット寮生へ伝えておきたいことは、できる限りぬいぐるみの交流を維持することと、ぬいぐるみの身体を大切にしてほしいということです。まず第一に汚れないようにすること、そしてその存在を忘れられないようにすること、そしてこうした交流する場を定期的に持つことによって自らの社会の中での存在を確かめること。それが大事なことだと考えております。

\talkernui{土偶}
土偶、です。やっぱりその、歴史を語るということは、語るという言葉の通り物語になってしまうところは一定あって、もちろん意識的に物語として歴史を構築することもありますけど、そうじゃなくても、我々が過去のことを思い浮かべるにあたって、我々としては当然そこにあるべきという分析単位を持ち込んでしまう、というのは非常に典型的かつ根源的な問題だと思います。ただその、歴史を紐解くという営みにおいて、今自分がいる場所というのがどのような歴史として語られる場所であるのかというのを想像するというのは、非常に有意義ですし、何より楽しいことだと思っています。今熊野寮が未来の皆さんから見てどのようなものになっているのか、というのを非常に楽しみにしています。ぜひまたたくさん話を聞かせてください。私からも話をしますのでよろしくお願いします。

\talkernui{リュカ}
いいまとめ。

\talkernui{カムイ}
このままぬいぐるみ座談会は終わりになります。僕の持ち主のやる気があったらまた来年もこういった企画をやるかもしれないので、そしたらまた集まって交流したいですね。

\talkernui{IKEAのサメ}
あとでみんなでしゅうごう写真とるサメ。

\talkernui{カムイ}
そうですね。あとで集合写真撮りましょう。ということで、ぬいぐるみ座談会は終わりです。お疲れ様でしたー!

\end{multicols}
\noindent{\uuline{\Large\textbf{編集後記\\}}}
ある夏の朝のこと。目が覚めると、カーテンの隙間から青々とした空が飛び込んできた。ソーダの中に、アイスクリームのような雲が浮かんでいる。自分もトッピングの一つになって、どこまでも飛んでいったらと想像すると、心が躍った。するとあまりの感動で気づかぬうちに身を乗り出していたのか、ベッドから落ちてしまった。起き上がってほこりを払い落し、再び窓の外を見る。草原に寝ころんで空を見上げ、涼しい風が毛を逆なでするのを身体全体で感じているかのようで、不思議と心地よい。しばらくすると持ち主が目を覚ましたのか、僕を持ち上げて布団に戻してくれた。きっと背中で何かを読み取ったのだろう。過ごしやすい気候になってからどこかにお出かけしようね、と言ってくれた。でもその前に寮内デビューしないとね、という話になり、それなら寮祭でぬいぐるみ座談会でもしようかということになった。
\par
当日は多くのぬいぐるみの方々が参加してくれて、とても楽しい会になった。後日ゲリラで催されたお茶会に招かれたりと、ぬいぐるみのコミュニティが広がっていくのは嬉しかった。持ち主も文字起こしは大変そうだったが、でも楽しかったと言っていた。また来年もやりたいな。
\par
\rightline{(文責 カムイ)}
\end{document}