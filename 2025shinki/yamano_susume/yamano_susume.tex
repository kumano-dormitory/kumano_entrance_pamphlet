\section{ 熊野寮的ヤマノススメ}


京都は盆地である。われわれが夏は暑く冬は寒い、お世辞にも過ごしやすいとは言えない環境に置かれているのもこの地形のためという。しかし疎ましい雨が地を潤し実りをもたらしているように、この世の大抵のものはデメリットばかりではなく、見方を変えてみればメリットもあるものだ。盆地とはすなわち平地が山に囲まれた地形であり、「山が近い」ということは山を登る人にとってはメリットとなる。自転車やバスで気軽に行ける範囲に、半日で登れる風光明媚で古くから信仰の対象ともなってきた山がいくつもあるのは京都ならではのことだ。どうせ京都に住むのなら、山を疎むのではなく楽しんでしまおうというのが、この記事の趣旨である。以下では初心者向けのおすすめの山について、寮からのアクセスと筆者のお気に入りポイントを解説する。なお、筆者は登山部でもなんでもなく年に数回ハイキングするだけの素人であるので、専門的な知識はないがあしからず。そして注意点として、どんな低い山にも事故のリスクはあるということを心に留めておいてほしい。体調の悪い時は登らない、動きやすい服装をする(長袖長ズボンがのぞましい)、十分な水と食料を持ち込む、日のあるうちに余裕を持って下山できるスケジュールを組むなど、誰にでもできる対策で構わないので安全には配慮していただきたい。

\subsection{①大文字山 標高:472m}
五山の送り火で大の字が点火されることで知られる。テレビで一度は見たことがあると思う。火床に足を踏み入れられるのと、思い立ったらすぐ登れるくらい簡単なのが良いところ。
寮からは自転車で10分ほどで、登山口はいくつかあるがどのルートでも登りやすいことに変わりはないので好きに選ぼう。自転車の場合は、ふもとの怒られなさそうなところに停めておこう。車の邪魔にならないかつ私有地ではない場所を選べばそうそうしょっぴかれないはずである。大勢で登りたければ、毎年文化部が開催する大文字コンパに参加しよう。
\subsection{②比叡山 標高:848m}
天台宗・最澄とセットで覚えさせられた、あの山である。滋賀県との県境に位置し、じつは山頂は滋賀県側なので延暦寺は京都のものではない。山登りのついでで延暦寺に参拝したり、琵琶湖を見下ろしたりできる一石三鳥な山。
寮からは自転車を使ってもいいが、登山口付近は住宅街でコインパーキングも停められそうな空き地もなく、叡山電鉄修学院駅横の駐輪場に停めることになるので、自転車で行っても結局駐輪代がかかるし登山口まで近いわけでもない。このため、出町柳駅から叡山電鉄に乗ることを推奨する。登山ルートは有名なきらら坂ルート(修学院駅)か、雲母坂ルートよりやや短い梅谷ルート(三宅八幡駅)の2つがおすすめ。それなりに登りがいのある山だが、大学生なら登山靴や杖がなくても普通の運動着で登り切れるくらいの手頃さである。ちなみにケーブルカーやドライブウェイが整備されているので、ハイキングをしなくても山頂に到達することができる。登っていて疲れ切ってしまったときは下りだけこれらに頼るのもアリ。
\subsection{③愛宕山(あたごやま) 標高:924m}
嵐山界隈に位置する、関西では火防の神として有名な愛宕神社がまします山。筆者のバイト先のキッチンにもここの御札が貼ってある。市街地からそこそこ離れていて、登山口ですら清らかな川を見ることができる。山頂の神社には手水舎がないので、参拝前に手を清めたい人はここの川の水を汲んでおこう。表参道はかなり広めの道幅が確保されていて、複数人で横に並んでも歩きやすい。丹波山地の南端に位置するため、北を見れば山海が広がり、南を見れば京都の市街地が見下ろせるのもおもしろい。
寮からはバスで一時間ほど揺られて向かう。熊野神社前や京都市役所前から嵐山方面のバスはそれなりに本数が出ているが、登山口である清滝停留所まで出る便は少ないので要チェック。標高は比叡山とそこまで変わらないが、筆者の主観的には比叡山よりも1.5倍ほどハードだった。登った時期の問題かもしれないが、高低による気温差が他の山よりも激しいように感じた。しかし厳しい分登ったときの達成感もひとしおであり、山頂の鳥居を見たときの喜びは忘れられない。実は細くはあるが山頂付近まで伸びる車道があり、ツーリングで来る人もいるらしい。バイク乗りの方は検討してみはいかがか。

文責:タケッシ