

\setlength{\parskip}{1em}
\setlength{\parindent}{0pt} 

\newtcolorbox{myboxnote}[1][]{%
  breakable           % ページを跨げるようにする
}


\begin{center}
    \Large\textbf{出張版C12談話室 on web 「とある日のこと」}\\
    \vspace{1em}
    \text{文責 : C12有志}
\end{center}

\vspace{1em}

\subsection{「とある日のこと」とは}
\noindent
 入寮パンフレットをパラパラと眺めているあなたの関心は、つまるところ「熊野寮にはどんな人が住んでいて、どのように暮らしているのか」にあるのではないでしょうか。\\
 本企画は、C12に住む寮生8名の、ある日の日記を集めたものです。切り出された出来事からはそれぞれの暮らしが、エピソードの選び方や筆致からはそれぞれの人柄が、ほのかに見えてくることと思います。\\
 人の数だけある生活、その断片を覗いてみてください。\\

\vspace{1em}
\begin{myboxnote}
\subsection{冬も愛おしく思えた一日}
\noindent
12月15日、12時半。昨日は部活の追いコンで朝まで飲んでいたため、当たり前のように昼に起きた。寒くて外に出る気力もないので、寮の食堂で卒論を書く。少し作業した後、談話室に寄ると、同じく食事中の同期Mがコタツに刺さっていた。C12談話室では最近テレビが解禁されて、日本のスナック文化について研究する外国人博士という、興味深い番組が流れていた。2人で「面白かったねー」なんて言いながら談話室を後にした。その後も引き続き作業を進め、今日は第1章の下書きができあがった。また談話室に戻って夕食を食べていると、先輩Aと後輩Sからパスタをおすそ分けしてもらうなどした。寮の委員会の仕事をして、風呂に入り、22時からブロック会議だった。長い会議が終わって、後輩Sの誘いに乗り、同期Mも一緒にファミマまで行って、2人でおでんを選んだ。談話室に帰って、駄弁ったり作業したりなんかしてると、車でラーメンに行ったはずの4人が早々に帰ってきた。墓地に迷い込んで、断念しちゃったんだって。私はというものの、おでんで大根とはんぺんしか食べなかったためお腹が空いて困っていた。これはちょうど良いと、ラーメンを食べ損ねて少ししょんぼりしている後輩Aをコンビニに誘った。コンビニの2往復なんて冷静に考えればしょうもないムーブのはずだけど、あの100mぐらいの道のりを、こうやって人と話しながら、あったかい食べ物を抱えて、てくてくと歩けるこの時間を今日はすごく幸せに思った。コンビニのおでんもかつ丼も、昔はしょっぱくて好みじゃなかったはずなのに、今ではすっかり慣れてしまった。それはきっと、寮生活で身に付いた一種の堕落であると同時に、みんなと過ごしてきた時間の積み重ねなんだと思う。ご飯もたくさん食べちゃったし、みんながゲームに興じてる姿を横目に見ながら、夜更かししようかな。真冬のコタツと談話室は、今日もほっこりとあたたかい。\\

\noindent
\textbf{うの}\\
総合人間学部4回生。1回生入寮。趣味は寮でのバンド活動(歌、ギター、ピアノ)。体育会サイクリング部所属。談話室では主に皆と話したり、バンドの練習をしています。

\vspace{1em}

\end{myboxnote}

\begin{myboxnote}
\subsection{241127 : よくものを失くす}
\noindent
 今日は無くしたものを拾ったと思ったら別の無くしたものに気がつく、そんな一日だった。\\
 先日、音源や過去に作成した楽曲のプロジェクトファイルがどかっと詰まった2TBのSSDを無くした。慌てふためくも用事のために泣く泣く電車に乗って数日間寮を空けていたが、この昼に自室に戻って机の下を探すと床に転がっていた。こんなところにいたのか。\\
 陽も暮れて夕方、C12の面々と紅葉を観に御所へと出かけた。正直なところ暗くてよく分からなかったが、わいわい出かけられたらそれで十分楽しいものだ、なんて思っていたら、寮に戻って一行で寮食をとろうという時に財布がないことに気がついた。一人でそそくさと輪を抜け、談話室や部屋を探すが見当たらない。血の気が引く。先ほどのお出かけで落としてしまったのかもしれない。\\
 小雨の降る中、一人で寒い道を歩く。スマホのライトを最大まで明るくし、見落としがないように注意深く探す。このときだけは御所のだだっ広さが憎かった。勝手な話だ。\\
 結局来るとこまで来てしまった。あとは帰り道を丹念に探すしかない。渋々交番に遺失届を出そうかと思ったが、その前にC12のラインで財布を無くした報告をすることにした。すると2分後、先輩が見つかったよとの連絡を下さった。もっと早めにSOSを出していたらこの無駄足はなかったかもとは思ったが、とにかく見つかったことにほっとした。\\
 寮に戻り、拾ってくれた人にお礼を言って財布を受け取り、寮食をとった。美味しかった。\\
 体を流して寝る準備をしようとシャワー室に行くと、今度はシャワーカードがなくなっていることに気がついた。冷や汗をかきながら事務室の落とし物入れを探していると、そこには見覚えのある私のシャワーカードが入っていた。胸を撫で下ろした。\\
 風呂上がり。乾燥が気になる季節だが化粧水も乳液もバイト先に忘れてしまった。シフトまでまだ日があるし、職場までは微妙に距離があって用事もなく行くのは面倒なので、その間はみっともなくバキバキな肌をマスクでなんとか隠していくことになりそうだ。あーあ。\\
 
\vspace{1mm}
\noindent
\textbf{白橋つむぐ}\\
音楽系の高校をフルート専攻生として卒業。今は経済学部で社会思想史を専攻している。

\end{myboxnote}
\vspace{1em}

\begin{myboxnote}
\subsection{訳ありレモン}
\noindent
 なんだか今日は息苦しい一日だった。無理に体をベッドから起こして、何のあてもなく徒歩一分のファミリーマートへ向かう。不穏な灰色の雲が、いつもよりも近くにある気がした。\\
 僕は閉塞感でいっぱいになっていた。自分も周りもすべて行き詰っているように思えてならず、何をしてもニヒリズムに陥りそうだった。逃げ出すように何かしたい、どこかに行きたいと一念発起するけれど、そんな元気なんてないぞ、と心の中の何かが言うとその決心は急速に萎んでしまうのだった。\\
 歩きながらふと、梶井基次郎の『檸檬』を思い出した。そういえば彼も晴れない憂鬱に頭を悩ませていたな、と妙な親近感を覚えた。が、それを解消するためだけにレモンを買いに行く気力も、かの有名な丸善がある新京極の雑踏を掻き分ける勇気もない自分にため息をついた。\\
 コンビニに着くと、怪しげに店内をうろついて物色した末、無駄に高い卵のパックだけを買って店を出た。改めて外から見る熊野寮は、要塞のようだ。僕の息苦しさはこの堅牢さから来ているのだろうか。自分の居室に帰っても気分が優れることはなさそうだったから、階段を上ってすぐの談話室に行くことにした。\\
 昼下がりの談話室には、まだ誰もいなかった。座って少しぼーっとしてから、さっき買った卵でだし巻き卵でも作ることにした。\\
 談話室から五歩も歩けば二階の炊事場がある。様々な人の生活感が溢れるその場所で、何やかんや混ぜて焼いて出来上がっただし巻きは、鮮やかな黄色だった。談話室に戻ってそれを食べながら眺めていると、また『檸檬』のことを思い出した。ふっくらとした黄色のだし巻きが、なんだか訳ありのレモンに見えてくる。熊野寮を木端微塵にするほどの爆発力はなさそうだ。けれどもその温かさと優しい食感が、僕の心を綻ばせていた。\\
 ふと外を見ると、空はさっきより明るくなって、窓には光が差し込んでいた。\\
 
\vspace{1mm}
\noindent
\textbf{まーてぃん}\\
理学部一回生。音楽が好き。よく聴くのは6,70年代のUKロック。\\
サイケデリックな生き方をしたい。

\end{myboxnote}
\vspace{1em}

\begin{myboxnote}
\subsection{12月16日}
\noindent
 『ドアを閉めて!イグアナが入ってきちゃう!』\\
 ホストマザーの叫ぶ声で目覚めた私。\\
 あ、これは夢じゃない、現実だ。昨日私は、マレーシアのホストファミリーのお家に到着したんだっけ。\\
 眠い目を擦りつつバタバタと支度し、タクシーに乗り込む。運転手さんは10歳年上のイスラーム教徒。私が日本人と分かると漫画のタイトルを列挙し始める。疎い分野だ。唯一分かったのはワンピース。\\
 気付けば目的地。初めて伺うお宅……初月忌で多くの人が集う。挨拶を終えた私は祭壇付近で様子を伺う。すると目の前の方が信仰するヒンドゥー教の神々の説明を始めて下さった。故人の弟さんだった。葬制の説明から身の上話まで、気付くとボイスレコーダーの記録が4時間を超えていた。\\
 電話が鳴る。前回の帰国時に見送って下さった女性の退院の知らせだった。急ぎ向かい、満面の笑みを浮かべる彼女と再会し胸を撫でおろす。\\
 19:30、隣人宅に移動し今週末のイベントで皆が披露する伝統舞踊の練習の見学......のはずが「見学」では済まなかった。\\
 20:00、逃げるように近くのヒンドゥー寺院に向かう。母の47回目の命日を迎える女性のお祈りの最中だった。私の姿を見るなり『まだ卒業できないのか!』と笑う司祭。\\
 20:30、オフィスで月末のイベントで振る舞う料理の打ち合わせ。日本ならお餅をつく時期、今年は朝から深夜までバナナの葉に囲まれるようだ。「ぐうぅ......。」隣の人のお腹が鳴る。気づけば22時過ぎ。『ケンタッキーが安いのは木曜か〜、新しくできた屋台行ってみる〜?あ、昨日の残りが冷蔵庫にあったわ〜』。\\
 帰るとほっとしたのか急にお腹が空く。緊張しながら少しいただいたお昼っきりか。スマホケースに挟んだままの熊野寮の食券。どんなに願ってもこれがいま高野豆腐に変わることはない。\\
 あぁ今日も、締切の迫る論文の査読の対応ができなかった。スマホを充電する余力もなく布団に潜り込む。\\
 でも充電切れでアラームが鳴らなくても心配はない。\\
 明日私を起こすのは、近所のモスクの礼拝の呼びかけか、それともボール片手に突撃してくる子供たちか。いや、熊野寮で迎える朝のようにニワトリかもしれない。\\

\vspace{1mm}
\noindent
\textbf{高野豆腐}\\
マレーシア語の発音の可愛さに魅せられ、早10年。直感で選んだ学部と院に続き、フィールドも初訪問でびびっときてから3年通い続ける。来年には博士号を取っているはず。好物の高野豆腐をカゴいっぱいに入れてレジの人を驚かせるのが好き。

\end{myboxnote}
\vspace{1em}

\begin{myboxnote}
\subsection{241219 : 寒いの嫌い寒いは寂しい}
\noindent
 冬が苦手だ。寒い空間がとびきり嫌い。だから、冬の寮はちょっぴり苦手。銭湯に行くのすら躊躇われる湯冷めの夜は、誰かと一緒にいないと寒さに覆われてしまう。\\
 昨夜はインターン先の同期と忘年会兼親睦会を終えて、川端通を北上していると0時を迎えていた。御池通の交差点で彼が私を待っていた。別れの挨拶をしてから、寮を通り越して彼の家に向かう。今日も一人で寝なくていい。私は安堵した。\\
 目覚めて玄関扉を開けると初雪に遭遇して、私は子供みたくはしゃいで見せた。今日は遠出の予定だったけど、何となくやめておこう。京都市内の飲食店を何軒か回って、最後に気になっていたバーに入る。彼が黙ると空気がやけに重い。今日はダメかもしれない。話し下手な私は一つのジョークも言えなかった。寒いと気が滅入る。\\
 カネコアヤノのヒットソングが虚空に流れて、二杯目にキューバリブレを頼む。キューバリブレと発語すると、陽気な餓鬼レンジャーの歌が頭の中に流れる。私を助けてくれ餓鬼レンジャー。\\
 …………ぐるぐる。気分を変えよう。早く帰ってこの日記を書こうと思った。\\
 寝床につきながら考える。今日は調子が悪いけど、自分の機嫌も取れないのに他人の機嫌を取ろうとしたのが間違いだった気がする。そもそも自分の気分の大波に悩まされる二十年だ。私は昆虫を夢中で追いかけているのが似合っている。そんな私が好きと彼も言う。冬は苦手だけど、またすぐに夏が来る。そんなふうに今日をやり過ごそう。明日は寒くないといいのに。\\

\vspace{1mm}
\noindent
\textbf{japonesa x}\\
メキシコ料理店でバイトしています。自他共に認めるナチョス狂。ビールと食らうと夏にぴったり!

\end{myboxnote}
\vspace{1em}

\begin{myboxnote}
\subsection{ローグ・ワン}

\noindent
「またね!」\\
「うん、またね!」\\
手を振って笑顔の彼女を見送る。\\
まただ。また伝えられなかった。\\
「今日こそは想いを伝えよう」そう思って約束を取り付けたのにまた何も言えなかった。嫌気がさす。いつもそうだ。博打を打っているように見せかけて勝てる勝負しかしたことがない。\\
後悔。自己嫌悪。鴨川沿いをトボトボと歩いて帰る。\\
風呂に入る。普段より長めにシャワーを浴びる。\\
布団に潜り込む。\\
スマホを開く。\\
スマホを閉じる。\\
スマホを開く。\\
スマホを閉じる。\\
布団を出る。\\
コーヒーを淹れる。\\
コーヒーを飲む。\\
布団に戻る。\\
またスマホを開く。\\
気づけばまた日が回っている。想いを自覚してから何日経った?覚悟を決めてから何日経った?\\
いつもはしないネットサーフィンをしてみる。\\
「告白 なんて言う」\\
「片思い 振られたら」\\
「彼氏持ち 告白」\\
「告白 勇気出る言葉 有名人」\\
そんな検索履歴が画面に並ぶ。中学生かよ。\\
 \\
「chat GPTに聞いてみるか?」自嘲的にそんな言葉を呟いてみる。\\
そういやこの前私が人を好きになったときには生成AIなんて言葉は誰も知らなかったよな。\\
「ガンバッテクダサイアナタナラデキマス!」「オウエンシテマス」「ユウキヲダシテ!」\\
うるさい。そんな言葉で想いを形にできたらこんなに苦労してない。\\
 \\
アプリを閉じてまたネットサーフィンを再開する。\\
「失敗しなくちゃ、成功はしないわよ」(ココ・シャネル)\\
時計を見る。3時半。\\
LINEを開く。\\
「言いたいことがあるので直近どこかで10分くらい時間もらえませんか」\\
それだけ送ってスマホを放り投げて目を瞑る。\\
なんでこんなしょーもない、この世の成功者が全員言ってそうな、ありきたりな”名言”で踏ん切りがついたんだ。我ながら単純な男だな。中学生かよ。\\
 \\
 \\
昨日とは逆向きに鴨川を歩く。\\
爆音で音楽を聴きながら歩く。\\
一人ヘドバンしながら歩く。\\
勝負の直前はいつもそう。\\
さ、待ち合わせ場所が見えてきた。昨日もここで会ったよな。\\


\vspace{1mm}
\noindent
\textbf{半分、ヒト}\\
Appare!というアイドルグループの「激奏!アンサンブル」という曲を爆音で聴き続けていました。\\
今年は当社比京都にいます。探し出して、是非仲良くしてあげてください。あと授業を一緒に受ける人探してます。

\end{myboxnote}
\vspace{1em}

\begin{myboxnote}
\subsection{241219:厄介事について}
\noindent
 最低気温が毎日0度近くになり、京都に本格的な寒さがやって来た。地元よりもずっと暖かいはずなのに、田舎と違って京都は車移動が普通ではないから外にいる時間が多く、その分だけ寒く感じる。バイト先で窓の外に目をやると、雪が激しく降っていた。\\
 帰宅時間になって建物を出るころ、雪はみぞれになっていた。ほかの人たちが傘をさしているのを見て、慌てて鞄のなかを探っても、折り畳み傘は入っていない。必要なときに準備を怠っていた過去の自分を恨みつつ、コートのポケットに手を突っ込み、身を縮こませながら駅まで歩いていくことにした。\\
 髪の毛が少しずつ濡れていくのを感じながら考えていたのは最近身の周りで起きた(起こされた、といった方がいいかもしれない)トラブルについてだった。先月も同様にトラブルに巻き込まれたので、二か月連続で厄介ごとを抱えていることになる。\\
 詳細は書けないのだが、変わった人、感覚のずれた人というのはどこにでもいるものだ、と最近思う。公務員になった友人たちからはそういった話をよく聞いている。人の話なら笑っていられるが、自分の身に降りかかるとそうはいかない。どうしても対応する過程で疲弊してしまう。困ったものだ。\\
 しかし、厄介ごとを引き起こす人が自分たちとは根本的に異質なわけではない------少なくとも自分はそう思っている。きっと、各々が自分の中に持っている感覚や常識のずれがこうしたトラブルや衝突を引き起こしてしまうのだ。皆が自分の目でしか世界を見ることはできないし、他人がどんな目で世界を見ているかはわからない。もしかすると、自分のそれもわかっていないのかもしれない。個々人にその常識を作ってきた特有の人生がある。陳腐な言葉で言えば、各々事情があるのだ。\\
 だからといって、別に人の気持ちを考えようとか、彼らに対して穏健な態度をとるべきだとか思っているわけでもない。厳しくない相手には増長してもよい、という常識を持っている人間だっている。毅然とした態度で相対しなければいけないことも多い。他人の事情より自分の事情である。\\
 そんなことを考えていると、ようやく駅にたどり着いた。とりあえずハンカチで濡れた頭を拭き、電車を待つ。電光掲示板を見ると、どうやら10分ほど遅延しているらしい。電車を待っている間、コートの左ポケットでぐっしょりとぬれたハンカチが手を冷やしていた。\\

\vspace{1mm}
\noindent
\textbf{荒砥}\\
大学院生\\
文学をやってます

\end{myboxnote}
\vspace{1em}

\begin{myboxnote}
\subsection{ありふれた夜を愛そう}
\noindent
 金曜の午後は毎週ゼミがあるのだが、この日は講義室で意見を交わす代わりに陳列館(史学系の文献をはじめ、東洋美術関連の図書や資料が収められている京大の施設)の掃除を研究室の学友たちとすることになっていた。院生の先輩の指示に従い、気の遠くなる作業をひとり黙々と進める中、資料室の向こうでは数人のグループになってカビ取りをしている学友たちの楽しそうな話し声が聞こえてくる。遅刻して行ったことで会話の集団からあぶれた自分は、ゼミでの肩身の狭さを感じながら陳列館の冷たい床につま先立ちして作業を続けるのだった。\\
 退屈な掃除もひと段落つき、打ち上げと忘年会を兼ねて研究室でパーティーが開かれた。クリスマスが近いこともあり、自分は自作のシュトーレンの余りを供出することで少しでも肩身の狭さを払拭してから打ち上げに加わろうと画策した。その企みは見事に功を奏し、自然と会話の輪の中に加わることができて、寮の話をする流れにもなった。\\
 しかし、寮のことを話していても、一応理解したように振舞ってはくれるがどこか一線を引かれているような感覚がどうしても抜けなかった。今でも吉田寮には入寮できること、ガサが不当であること、タテカンは立てれば立てるだけ良いということ、ひとつひとつの固い結び目をほどくように根気よく話してはみるが、なかなか芯まで伝わっている実感がない。無力感に襲われて机に転がったほろよいの空き缶を手で潰してみたりしているうちに、気がつけば話題は年末の予定などに移ってしまっていた。せっかく馴染めたと思った輪の中で自分だけ酔いがさめてしまった感じがした。\\
 どこか消化不良な気分で研究室を後にし、人も車もまばらな東大路の緩やかな下り坂に身を任せてゆっくりと、いつものように寮に帰る。そのまま部屋に戻って横になっても良かったが、今は少し人と喋りたいと思って食堂に向かった。食堂では仲のいい同期や先輩が机を囲んで談話しており、えもいえぬ安心感というか、帰ってきたなあという実感を与えてくれる。\\
 ファミマで買ったおでんやカップそばを食べながら、食堂に残っていた5人ほどでYouTubeに転がっている適当なイントロクイズに興じる。フリー音源のイントロじゃ全然分からないと文句を言いながら、「kiroroっぽい」とか「シティーハンターのオープニングっぽい」とか適当なことを言っては共感したりされたりして笑い合っていた。いろんなことでモヤモヤしたり考え過ぎたりすることもあるけど、こういう何気ない夜に、何気ない会話に何度も救われているなと思うし、改めて自分は寮のことが好きなんだなと実感した日だった。\\

\vspace{1mm}
\noindent
\textbf{豆板醤}\\
美術史を学んでいる。人と話すのが好き。

\end{myboxnote}
\vspace{1em}

\subsection{おわりに}
\noindent
本企画は、タイトルの通り「C12談話室 on web」の出張版です。3ヶ月に一度更新する「季刊C12」を始めとして、C12に関わりのある人々がいろいろな文章を寄せておりますので、よろしければ覗いてみてください。\\

\begin{figure}[ht]
\includegraphics[scale=0.1, bb=0 0 500 500]{2025shinki/c12_danwa_one_day/kumanoc12com.jpg}
\end{figure}

